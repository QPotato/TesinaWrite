\chapter*{Resumen}
\addcontentsline{toc}{chapter}{Resumen}


El proceso de aprendizaje de juego ficticio, propuesto por primera vez por Brown en 1951 \cite{brown:1951} como un método para encontrar equilibrios a través de la repetición de un juego en un proceso iterativo, nos provee una posible explicación de como un grupo de jugadores racionales pueden llegar a un equilibrio de Nash. Brown propuso dos variantes de juego ficticio: simultáneo y alternante, que se diferencian en la información que en cada iteración tienen los jugadores al momento de tomar su decisión.

Desde la teoría de juegos algorítmica, la utilidad del juego ficticio fue cuestionada en publicaciones que argumentan que en los casos que converge, su velocidad de convergencia puede ser muy inferior a la de metodos alternativos para encontrar equilibrios de Nash, como las mecánicas de no arrepentimiento o la resolución del problema de optimización lineal equivalente \cite{modified:fp:linear}. Sin embargo, estos trabajos se refieren solo al juego ficticio simultáneo y no existen en la literatura actual estudios sobre la velocidad de convergencia del juego ficticio alternante y su comparación contra la variante simultánea.

En este trabajo, extenderemos el estudio realizado en 2013 por Brandt, Fischer y Harrenstein \cite{brandt:rate:convergence} sobre la velocidad de convergencia del juego ficticio simultáneo a la variante alternante y aportaremos algunos resultados que indican que esta podría tener una utilidad práctica superior a su contraparte simultánea como mecanismo para encontrar equilibrios de Nash.