\documentclass[pdf]{beamer}
\usetheme{Copenhagen}
\mode<presentation>{}

\usepackage{pgfpages}
%\setbeameroption{show notes on second screen=right}
%\setbeameroption{show only notes}

\usepackage[utf8]{inputenc}
\usepackage{babel}
\usepackage{paralist}
\usepackage{todonotes}
\usepackage{stmaryrd}
\usepackage{graphicx,xcolor}
\usepackage{subcaption}
\graphicspath{{Imgs/}}

\usepackage{array}


\newcommand{\pstrat}{\widetilde}
\DeclareMathOperator*{\argmax}{argmax}

\title{Game Theory and Fictitious Play}

\author[]{Federico Badaloni}

\subject{Tesina}

\usepackage{color}

\begin{document}

\begin{frame}
    \frametitle{About this presentation}
    \begin{itemize}
        \item Sacrificed Math rigurosity in in favor of fun.
        \item This is practice for my thesis defense. Remember to record.
        \item Didn't get to fit everything I wanted. Making slides is hard.
        \item \pause Remember to turn pauses on.
    \end{itemize}

\end{frame}

\begin{frame}
    \frametitle{Example: Fede's Dilemma}
    An original example invented by me. You have never heard of it before.

\end{frame}

\begin{frame}
    \frametitle{General definition}
    \textbf{Our problem}: modeling situations where players try to maximize their co-dependent scores by choosing from their respective
    sets of actions in a rational, selfish simultaneous and independent way.
    
    \pause
    \begin{definition}[Strategic Game]
        A strategic game is a tuple $(P, (A_p)_{p \in P}, (u_p)_{p \in P})$ where:
        \begin{itemize}
            \item P is the set of players.
            \item $\forall p \in P$, $A_p$ is the set of actions of player $p$. Elements of $\prod\limits_{p \in P} A_p$ are called action profiles.
            \item $\forall p \in P$, $u_p: \prod\limits_{\widehat{p} \in P} A_{\widehat{p}} \rightarrow \mathbb{R}$ is the utility function or payoff function of player $p$.
        \end{itemize}
    \end{definition}

\end{frame}

\begin{frame}
    \frametitle{Fede's Dilemma with the general definition}
    \begin{itemize}
        \item $P = \{f, b\}$
        \item \pause $A_f = A_b = \{Confess, Silence\}$
        \item \pause $u_f: A_f \times A_b \rightarrow \mathbb{R}$
        \begin{itemize}
            \item $u_f(S, S) = -1$
            \item $u_f(S, C) = -3$
            \item $u_f(C, S) = 0$
            \item $u_f(C, S) = -2$
        \end{itemize}
    \end{itemize}

\end{frame}

\begin{frame}
    \frametitle{Games in bi-matrix form}
        \textbf{Smaller scope}: we'll focus on 2 player games. We can represent them with a cool matrix of tuples $(A, B)$ like this:

        \begin{center}
    \begin{tabular}{lllll}
                                 & $j_1$                 & $j_2$                  & $\cdots$ & $j_m$                 \\ \cline{2-5}
    \multicolumn{1}{l|}{$i_1$}   & $(a_{1,1}, b_{1,1})$  & $(a_{1,2}, b_{1,2})$   & $\cdots$ & \multicolumn{1}{l|}{$(a_{1,2}, b_{1,m})$}  \\
    \multicolumn{1}{l|}{$i_2$}   & $(a_{2,1}, b_{2,1})$  & $(a_{2,2}, b_{2,2})$   & $\cdots$ & \multicolumn{1}{l|}{$(a_{2,m}, b_{2,m})$}  \\
    \multicolumn{1}{l|}{$\vdots$}& $\vdots$              & $\vdots$               & $\ddots$ & \multicolumn{1}{l|}{$\vdots$   }           \\
    \multicolumn{1}{l|}{$i_n$}   & $(a_{n,1}, b_{n,1})$  & $(a_{n,2}, b_{n,2})$   & $\cdots$ & \multicolumn{1}{l|}{$(a_{n,m}, b_{n,m})$}  \\ \cline{2-5}
    \end{tabular}
\end{center}

        \begin{itemize}
            \item Player 1 (row player) has actions $i \in N = \{i_1, i_2, \dots, i_n\}$.
            \pause
            \item Player 2 (column player) has actions $j \in M = \{j_1, j_2, \dots, j_m\}$.
            \pause
            \item $A, B \in \mathbb{R}^{n \times m}$ are called the payoff patrices.
            \pause
            \item If player 1 plays $i$ and player 2 plays $j$, the payoff for player 1 will be $a_{i,j}$ and the payoff for player 2 will be $b_{i,j}$
        \end{itemize}
\end{frame}

\begin{frame}
    \frametitle{Fede's Dilemma in bi-matrix form}
    \begin{center}
    \begin{tabular}{lll}
                                 & Silence     & Confess                         \\ \cline{2-3}
    \multicolumn{1}{l|}{Silence}   & $(-1, -1)$  & \multicolumn{1}{l|}{$(-3, 0)$}  \\
    \multicolumn{1}{l|}{Confess}   & $(0, -3)$  & \multicolumn{1}{l|}{$(-2, -2)$}  \\ \cline{2-3}
    \end{tabular}
\end{center}
\end{frame}

\begin{frame}
    What if players are not deterministic?
    \begin{itemize}
        \item A mixed strategy for player 1 is a probability distribution over $N$, represented as a row vector $x \in \Delta(N)$ of length $n$. A mixed strategy that assigns probability $1$ to action $i \in N$ is represented as $\pstrat{i}$ and is called a pure strategy. The expected payoffs for each action of player 2 against a mixed strategy $x$ of player $1$ are $xB$.
        \pause
        \item A mixed strategy for player 2 is a probability distribution over $M$, represented as a column vector $y \in \Delta(M)$ of length $m$. The expected payoffs for player 1 against a mixed strategy $y$ of player $2$ are  $Ay$
        \pause
        \item If player 1 plays the mixed strategy $x$ and player 2 plays the mixed strategy $y$ then expected payoffs for each player are $xAy$ and $xBy$ respectively. 
    \end{itemize}
\end{frame}

\begin{frame}
    \frametitle{Fede's Dilemma: payoffs}
    
    Spreadsheet


\end{frame}

\begin{frame}
    \frametitle{Best Response and Nash Equilibrium}

    \begin{itemize}
        \item We define the Best Response Set for player 1 to player 2's mixed strategy $y$ as $BR_1(y) = \argmax_{i \in N}\{ \pstrat{i}Ay\}$
        \pause
        \item Similarly, the Best Response Set for player 2 to player 1's mixed strategy $x$ is $BR_2(x) = \argmax_{j \in M}\{ xB\pstrat{j}\}$
        \pause
        \item A pure Nash Equilibrium is a profile of pure strategies $(\pstrat{i^*}, \pstrat{j^*})$ such that $i^* \in BR_1(\pstrat{j^*})$ and $j^* \in BR_2(\pstrat{i^*})$
        \item \pause $(C, C)$ is a Nash Equilibirum and we both knew it.
    \end{itemize}
\end{frame}

\begin{frame}
    \frametitle{New Example}
    Clip from a 90s cartoon
    \pause \begin{center}
    \begin{tabular}{llll}
                                 & $j_R$    & $j_P$     & $j_S$                         \\ \cline{2-4}
    \multicolumn{1}{l|}{$i_R$}   & $0$      & $-1$      & \multicolumn{1}{l|}{$1$}  \\
    \multicolumn{1}{l|}{$i_P$}   & $1$      & $0$       & \multicolumn{1}{l|}{$-1$}  \\
    \multicolumn{1}{l|}{$i_S$}   & $-1$     & $1$       & \multicolumn{1}{l|}{$0$}  \\ \cline{2-4}
    \end{tabular}
\end{center}
    \begin{itemize}
        \item \pause Bart and Lisa have played this game many times before in an iterative process.
        \item \pause Bart has a constant pure Rock strategy.
        \item \pause Lisa is doing something smarter. She's playing the Best Response to what she perceives as Bart's strategy over the previous iterations of the game. She knows that $BR_1(\pstrat{j_R}) = \{ i_P\}$ so she plays Paper. 
        \item \pause What happen's if we make Lisa play against herself?
    \end{itemize}
\end{frame}

\begin{frame}
    \frametitle{Fictitious Play}

    \begin{definition} \label{def:fp:berger}
        Given a game $(A, B)$  of size $n \times m$, a sequence of action profiles $(i^\tau, j^\tau)$, and a pair of sequences $x^\tau$ and $y^\tau$ (called belief sequences) such that for every $\tau \in \mathbb{N}$:
        \begin{gather*}
            x^\tau= \frac{\sum^\tau_{s=1} \pstrat{i^s}}{\tau}  \\
            y^\tau= \frac{\sum^\tau_{s=1} \pstrat{j^s}}{\tau}
        \end{gather*}
        Then $(i^\tau, j^\tau)$ is a Fictitious Play sequence if $(i^1, j^1)$ is an arbitrary element of $N \times M$ and for every $\tau \in \mathbb{N}$ it's true that $i^{\tau+1} \in BR_1(y^\tau)$ and $j^{\tau+1} \in BR_2(x^\tau)$.
    \end{definition}
    Each player consider each other's history as a mixed strategy and play a best response each round.
\end{frame}

\begin{frame}
    \frametitle{Fictitious Play: code}
    \begin{itemize}
        \item Show RPS in code.
    \end{itemize}    

\end{frame}

\begin{frame}
    \frametitle{The Stability Principle}
    \begin{theorem}[Stability Principle]
        Given a Fictitious Play sequence $(i^\tau, j^\tau)_{\tau \in \mathbb{N}}$ in $(A, B)$ with belief sequences $x^\tau$ and $y^\tau$. If $(i^k, j^k)$ is a Nash Equilibrium then $(i^{k+1}, j^{k+1}) = (i^k, j^k)$.
    \end{theorem}
    \pause If the process reaches a Nash Equilibrium then it'll keep playing that Nash Equilibrium.

    \pause \textbf{Intuition}: if the process plays some action profile, then each players incentives to play the best response against their oponents action raises. But if the profile is a Nash Equilibrium then the best response is the action they just played.
\end{frame}

\begin{frame}
    \frametitle{Proof of the Stability Principle}
    \begin{lemma}[1]
        Given a Fictitious Play sequence $(i^\tau, j^\tau)_{\tau \in \mathbb{N}}$ with belief sequences $x^\tau$ and $y^\tau$. If $(i^k, j^k)$ is a Nash Equilibrium then $i^k \in BR_1(y^{k})$ and $j^k \in BR_2(x^{k+1})$.
    \end{lemma}
\end{frame}

\begin{frame}
    \frametitle{Proof of the Stability Principle}
    \begin{align*}
        BR_1(y^k) &= BR_1(\frac{k - 1}{k} y^{k-1} + \frac{ \pstrat{j^k}}{k}) \\
        &= \argmax_{i \in N} \{\pstrat{i} A(\frac{(k - 1)y^{k-1}}{k} + \frac{ \pstrat{j^k}}{k})\}\\
        &= \argmax_{i \in N} \{\frac{(k - 1)}{k}\pstrat{i} Ay^{k-1} + \frac{1}{k}\pstrat{i} A \pstrat{j^*}\}
    \end{align*}
    \begin{itemize}
        \item \pause The definition of Fictitious Play implies that $i^k \in BR_1(y^{k-1}) = \argmax_{i \in N}\{\pstrat{i}Ay^{k-1}\}$.
        \item \pause The definition of Nash Equilibrium implies that $i^k \in BR_1(\pstrat{j^k}) = \argmax_{i \in N}\{\pstrat{i} A \pstrat{j^k}\}$.
        \item \pause $\frac{(k - 1)}{k}\pstrat{i}Ay^{k-1} + \frac{1}{k}\pstrat{i} A \pstrat{j^k}$ is just a linear combination.
        \item \pause Analogous for player 2.
    \end{itemize}
\end{frame}

\begin{frame}
    \frametitle{Proof of the Stability Principle}
    \begin{lemma}[2]
        Given a Fictitious Play sequence $(i^\tau, j^\tau)_{\tau \in \mathbb{N}}$ with belief sequences $x^\tau$ and $y^\tau$. If $(i^k, j^k)$ is a Nash Equilibrium then $BR_1(y^{k}) \subseteq BR_1(y^{k-1})$ and $BR_2(x^{k+1}) \subseteq BR_2(x^{k})$.
    \end{lemma}
    \pause If the process plays a Nash equilibrium then there are no new actions in the best response sets in the next iteration.
\end{frame}

\begin{frame}
    \frametitle{Proof of the Stability Principle}
    \begin{itemize}
        \item We can prove $BR_1(y^{k}) \subseteq BR_1(y^{k-1})$ by proving $i \notin BR_1(y^{k-1}) \implies i \notin BR_1(y^{k})$.
        \item \pause If $i^k \in BR_1(y^{k-1})$ but $i' \notin BR_1(y^{k-1})$ then $\pstrat{i^k}Ay^{k-1} > \pstrat{i'}Ay^{k-1}$
        \item \pause $(i^k, j^k)$ is a Nash Equilibrium so $\pstrat{i^k}A\pstrat{j^k} \ge \pstrat{i'}A\pstrat{j^k}$.
    \end{itemize}
    \pause
    \begin{align*}
        \pstrat{i^k}Ay^k &= \pstrat{i^k}A(\frac{k - 1}{k} y^{k-1} + \frac{\pstrat{j^k}}{k}) = \frac{(k - 1)}{k} \pstrat{i^k}Ay^{k-1} + \frac{1}{k}\pstrat{i^k}A\pstrat{j^k}\\
        &> \frac{(k - 1)}{k} \pstrat{i'}Ay^{k-1} + \frac{1}{k}\pstrat{i'}A\pstrat{j^k} = \pstrat{i'}A(\frac{k - 1}{k} y^{k-1} + \frac{\pstrat{j^k}}{k}) = \pstrat{i'}Ay^k
    \end{align*}
    \pause If the expected payoff for $i^k$ is greater than for $i'$ and $i^k$ is in the best response set, then $i'$ is not.
    
    \begin{quote}[Player 2]
        omg, same!
    \end{quote}
\end{frame}

\begin{frame}
    \frametitle{Rate of convergence: Example}
    How fast is Fictitious Play as an algorithm to find Nash Equilibriums?

    \pause Consider this game with $\epsilon = 2^{-k}$:
    \begin{center}
    \begin{tabular}{llll}
    .                          & $b^1$    & $b^2$                      & $b^3$                           \\ \cline{2-4}
    \multicolumn{1}{l|}{$a^1$} & $(1, 1)$ & $(2, 2)$                   & \multicolumn{1}{l|}{$(0, 0)$} \\
    \multicolumn{1}{l|}{$a^2$} & $(0, 0)$ & $(2+\epsilon, 2+\epsilon)$ & \multicolumn{1}{l|}{$(3, 3)$} \\ \cline{2-4}
    \end{tabular}
\end{center}
    
    \begin{itemize}
        \item \pause $(i_2, j_3)$ is the only Nash Equilibrium. We can prove this by iterated elimination of strictly dominated action (jargon for "discard bad choices")
        \item \pause This game can be codified in $O(k)$ bits.
        \item \pause If we start a Fictitious Play process with $(i_1, j_1)$, the sequence will play $2^k$ iterations before playing $(i_2, j_3)$.
        \item \pause For some reason I can't render tables in beamer, so I'll show them in my thesis. Ignore the spanish.
    \end{itemize}
    
\end{frame}

\begin{frame}
    \frametitle{Alternating Fictitious Play}
    \begin{definition} \label{def:fp:berger}
        Given a game $(A, B)$  of size $n \times m$, a sequence of action profiles $(i^\tau, j^\tau)$, and a pair of sequences $x^\tau$ and $y^\tau$ (called belief sequences) such that for every $\tau \in \mathbb{N}$:
        \begin{gather*}
            x^\tau= \frac{\sum^\tau_{s=1} \pstrat{i^s}}{\tau}  \\
            y^\tau= \frac{\sum^\tau_{s=1} \pstrat{j^s}}{\tau}
        \end{gather*}
        Then $(i^\tau, j^\tau)$ is an Alternating Fictitious Play sequence if $i^1$ is an arbitrary element of $N \times M$ and for every $\tau \in \mathbb{N}$ it's true that $i^{\tau+1} \in BR_1(y^\tau)$ and $j^{\tau} \in BR_2(x^\tau)$.
    \end{definition}
    \pause We break the principle of simultaneous decision making.

\end{frame}

\begin{frame}
    \frametitle{Alternating Fictitious Play in the example}

    For both starting actions of the row player, AFP converges in just 4 moves, independently of $\epsilon$.
    

\end{frame}

\end{document}
