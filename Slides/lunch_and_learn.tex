\documentclass[pdf]{beamer}
\usetheme{Copenhagen}
\mode<presentation>{}

\usepackage{pgfpages}
%\setbeameroption{show notes on second screen=right}
%\setbeameroption{show only notes}

\usepackage[utf8]{inputenc}
\usepackage{babel}
\usepackage{paralist}
\usepackage{todonotes}
\usepackage{stmaryrd}
\usepackage{graphicx,xcolor}
\usepackage{subcaption}
\graphicspath{{Imgs/}}

\usepackage{array}

\title{Game Theory and Fictitious Play}

\author[]{Federico Badaloni}

\subject{Tesina}

\usepackage{color}

\begin{document}

\begin{frame}
\frametitle{Summary}

    \begin{itemize}
    \item Part 1: Introduction to Game Theory.
    \item Part 2: The Fictitious Play algorithm. The Stability Principle.
    \item Part 3: Alternating Fictitious Play.
    \item Part 4: Convergence rates.
    \end{itemize}

\end{frame}

\begin{frame}
    \frametitle{Game Theory 1: definitions}
    - Informal description
    - More formally: (tuples)
    - Actions, Mixed Strategies
    - Alternatively: bimatrix games
    - payoffs as matrix products
\end{frame}

\begin{frame}
    \frametitle{Game Theory 2: examples}
    - Rock, Paper, Sicssors.
    - Prisonner's Dilemma.
\end{frame}

\begin{frame}
    \frametitle{Game Theory 3: moar definitions}
    - Best Response Set
    - Nash Equilibrium
    - back to examples
\end{frame}

\begin{frame}
    \frametitle{Fictitious Play 1: example}
    - Bart vs Lisa (explain Lisa)
    - Lisa vs Lisa
\end{frame}

\begin{frame}
    \frametitle{Fictitious Play 2: definition}
    - Berger Definition, SFP
    - Prisonner's Dilemma
    - Stability Principle
\end{frame}

\begin{frame}
    \frametitle{Fictitious Play 3: proof of stability principle}
    - Lemma 1
    - Lemma 2
    - Theorem 1
\end{frame}

\begin{frame}
    \frametitle{Alternating Fictitious Play}
    - Berger definition
    - Prissoner's Dilemma
    - RPS
\end{frame}

\begin{frame}
    \frametitle{Rate of convergence}
    - Theorem 4
\end{frame}

\begin{frame}
    \frametitle{Code!}
    - Implementation
    - Prissoner's Dilemma
    - Rock Paper Sicssors
    - Theorem 4
\end{frame}
\end{document}
