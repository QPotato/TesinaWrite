\documentclass[handout, pdf]{beamer}
\usetheme{Copenhagen}
\mode<presentation>{}

\usepackage{pgfpages}
%\setbeameroption{show notes on second screen=right}
%\setbeameroption{show only notes}

\usepackage[utf8]{inputenc}
\usepackage[spanish]{babel}
\usepackage{paralist}
\usepackage{todonotes}
\usepackage{stmaryrd}
\usepackage{graphicx,xcolor}
\usepackage{subcaption}
\graphicspath{{Imgs/}}

\usepackage{array}


\newcommand{\pstrat}{\widetilde}
\DeclareMathOperator*{\argmax}{argmax}

\title{Velocidad de convergencia del juego ficticio alternante}

\author[]{Federico Badaloni}

\subject{Tesina}

\usepackage{color}

\begin{document}

\begin{frame}
    \frametitle{Juegos en forma matricial}
        \begin{itemize}
            \item Representaremos juegos de 2 jugadores con un par de matrices $(A, B)$ donde $A, B \in \mathbb{R}^{n \times m} $
            \begin{center}
    \begin{tabular}{lllll}
                                 & $j_1$                 & $j_2$                  & $\cdots$ & $j_m$                 \\ \cline{2-5}
    \multicolumn{1}{l|}{$i_1$}   & $(a_{1,1}, b_{1,1})$  & $(a_{1,2}, b_{1,2})$   & $\cdots$ & \multicolumn{1}{l|}{$(a_{1,2}, b_{1,m})$}  \\
    \multicolumn{1}{l|}{$i_2$}   & $(a_{2,1}, b_{2,1})$  & $(a_{2,2}, b_{2,2})$   & $\cdots$ & \multicolumn{1}{l|}{$(a_{2,m}, b_{2,m})$}  \\
    \multicolumn{1}{l|}{$\vdots$}& $\vdots$              & $\vdots$               & $\ddots$ & \multicolumn{1}{l|}{$\vdots$   }           \\
    \multicolumn{1}{l|}{$i_n$}   & $(a_{n,1}, b_{n,1})$  & $(a_{n,2}, b_{n,2})$   & $\cdots$ & \multicolumn{1}{l|}{$(a_{n,m}, b_{n,m})$}  \\ \cline{2-5}
    \end{tabular}
\end{center}
            \pause
            \item El jugador 1 (jugador fila) tiene acciones $i \in N = \{i_1, i_2, \dots, i_n\}$.
            \pause
            \item El jugador 2 (jugador columna) tiene acciones $j \in M = \{j_1, j_2, \dots, j_m\}$.
            \pause
            \item Si el jugador 1 juega $i$ y el jugador 2 juega $j$, la ganancia del jugador 1 será $a_{i,j}$ y la ganancia del jugador 2 será $b_{i,j}$
            \pause
            \item $A$ y $B$ son llamadas matrices de pagos.
        \end{itemize}
\end{frame}

\begin{frame}
    \frametitle{Equilibrio de Nash (NE) puro}
        \begin{itemize}
        \item Es un perfil $(i, j)$ tal que $a_{i,j} \geq a_{i', j} \forall{i' \neq i} \land b_{i,j} \geq b_{i, j'} \forall{i' \neq i}$.
        \pause
        \item $a_{i, j}$ es un máximo de fila en A.
        \pause
        \item $b_{i, j}$ es un máximo de columna en A.
        \pause
        \item Es decir, es un perfil tal que ninguno de los jugadores tiene un incentivo para cambiar su jugada.
        \end{itemize}
\end{frame}

\begin{frame}
    \frametitle{Juego degenerado}
        \begin{itemize}
        \item Decimos que un juego $(A, B)$ es degenerado si $(\exists i_1, i_2,j / i_1 \neq i_2 \land a_{{i_1}, j} = a_{{i_2}, j}) \lor (\exists i, j_1, j_2/ j_1 \neq j_2 \land b_{i, {j_1}} = b_{i, {j_2}})$ 
        \item Es decir, para alguna de ambas matrices en alguna fila o columna se repite algún valor.
        \end{itemize}
\end{frame}

\begin{frame}
    \frametitle{Familias de juegos}
    Un juego $(A, B)$ es  
        \begin{itemize}
        \item de intereses idénticos  si $A = B$.
        \item de suma cero si $A = -B$.
        \item simétrico si es de suma cero y además $A = -A^t$ y $B = -B^t$. 
        \end{itemize}
    En todos estos casos, podemos describir el juego dando solo la matriz $A$.
\end{frame}

\begin{frame}
    \frametitle{Estrategias Mixtas}
    \begin{itemize}
        \item Una estrategia mixta para el jugador 1 es una distribución de probabilidades sobre $N$, que representaremos con un vector $x \in \Delta(N)$ de tamaño $n$.
        \pause
        \item Una estrategia mixta para el jugador 2 es una distribución de probabilidades sobre $M$, que representaremos con un vector $y \in \Delta(M)$ de tamaño $m$.
        \pause
        \item A las estrategias mixtas que asignan probabilidad $1$ a una acción las llamaremos estrategias puras y las notaremos con $\pstrat{i}$ y $\pstrat{j}$.
        \end{itemize}
\end{frame}

\begin{frame}
    \frametitle{Ganancias esperadas}
    Podemos expresar algebraicamente el valor esperado de las ganancias de un jugador contra una estrategia mixta del otro.
    \pause
    \begin{itemize}
        \item La ganancias esperadas del jugador $1$ contra una estrategia mixta $y$ del jugador $2$ son $Ay$.
        \pause
        \item La ganancias esperadas del jugador $2$ contra una estrategia mixta $x$ del jugador $1$ son $xB$.
        \pause
        \item Si el jugador 1 juega la estrategia mixta $x$ y el jugador 2 juega la estrategia mixta $y$ entonces sus ganancias esperadas serán $xAy$ y $xBy$ respectivamente.
        \pause
        \item Las estrategias puras proyectan el vector de ganancias esperadas a una acción. 
    \end{itemize} 
\end{frame}

\begin{frame}
    \frametitle{Mejor respuesta}

    \begin{itemize}
        \item Dada una estrategia del rival, la \emph{mejor respuesta} es un conjunto de acciones que maximizan la ganancia esperada.
        \item Definimos el conjunto de mejores respuestas del jugador 1 contra la estrategia mixta $y$ del jugador 2 como $BR_1(y) = \argmax_{i \in N}\{ \pstrat{i}Ay\}$
        \pause
        \item Similarmente, el conjunto de mejores respuestas del jugador 2 contra la estrategia mixta $x$ del jugador 1 es $BR_2(x) = \argmax_{j \in M}\{ xB\pstrat{j}\}$
        \pause
        \item Un equilibrio de Nash puro es un perfil de estrategias $(\pstrat{i^*}, \pstrat{j^*})$ tal que $i^* \in BR_1(\pstrat{j^*})$ y $j^* \in BR_2(\pstrat{i^*})$
    \end{itemize}
\end{frame}

\begin{frame}
    \frametitle{Reglas de desempate}
    Una regla de desempate es un par de funciones
        \begin{itemize}
        \item $d_1: P(N) \rightarrow N / d_1(S) \in S$
        \item $d_2: P(M) \rightarrow M / d_2(S) \in S$
        \end{itemize}
    tales que $d_1$ y $d_2$ cumplen la independencia de alternativas irrelevantes:
    \[
        d(T) = a \in S \subseteq T \implies d(S) = a
    \]

    \pause
    Esto equivale a tener un orden total para cada conjunto de acciones.
\end{frame}

\begin{frame}
    \frametitle{Juego ficticio simultaneo (SFP)}
    \begin{definition} 
        Dado un juego $(A, B)$  de tamaño $n \times m$, una secuencia de perfiles de acciones $(i^\tau, j^\tau)$, y un par de secuencias $x^\tau$ y $y^\tau$ (llamadas secuencias de creencias) tales que para cada $\tau \in \mathbb{N}$:
        \begin{gather*}
            x^\tau= \frac{\sum^\tau_{s=1} \pstrat{i^s}}{\tau}  \\
            y^\tau= \frac{\sum^\tau_{s=1} \pstrat{j^s}}{\tau}
        \end{gather*}
        Entonces $(i^\tau, j^\tau)$ es una secuencia de juego ficticio simultaneo si $(i^1, j^1)$ es un elemento arbitrario de $N \times M$ y para todo $\tau \in \mathbb{N}$, $i^{\tau+1} \in BR_1(y^\tau)$ y $j^{\tau+1} \in BR_2(x^\tau)$.
    \end{definition}
    Cada jugador considera el historial del otro como una estrategia mixta y juega la mejor respuesta en cada ronda.
\end{frame}

\begin{frame}
    \frametitle{Juego ficticio alternante (AFP)}
    \begin{definition} 
        Dado un juego $(A, B)$  de tamaño $n \times m$, una secuencia de perfiles de acciones $(i^\tau, j^\tau)$, y un par de secuencias $x^\tau$ y $y^\tau$ (llamadas secuencias de creencias) tales que para cada $\tau \in \mathbb{N}$:
        \begin{gather*}
            x^\tau= \frac{\sum^\tau_{s=1} \pstrat{i^s}}{\tau}  \\
            y^\tau= \frac{\sum^\tau_{s=1} \pstrat{j^s}}{\tau}
        \end{gather*}
        Entonces $(i^\tau, j^\tau)$ es una secuencia de juego ficticio alternante si $i^1$ es un elemento arbitrario de $N$ y para todo $\tau \in \mathbb{N}$, $i^{\tau+1} \in BR_1(y^\tau)$ y $j^{\tau} \in BR_2(x^\tau)$.
    \end{definition}
    \pause La toma de decisiones ya no es simultanea e independiente.

\end{frame}

\begin{frame}
    \frametitle{El principio de estabilidad}
    \begin{theorem}[Principio de estabilidad]
        Dada una secuencia de juego ficticio $(i^\tau, j^\tau)_{\tau \in \mathbb{N}}$ en $(A, B)$ con secuencias de creencias $x^\tau$ y $y^\tau$. Si $(i^k, j^k)$ es un equilibrio de Nash puro, entonces $(i^{k+1}, j^{k+1}) = (i^k, j^k)$.
    \end{theorem}
    \pause Si el proceso alcanza un equilibrio de Nash entonces seguirá jugando ese equilibrio de Nash.

    \pause \textbf{Intuición}: Si el proceso juega un perfil de acciones, entonces los incentivos de cada jugador para jugar la mejor respuesta contra la acción del oponente en ese perfil crece. Pero si el perfil es un equilibrio de Nash entonces la mejor respuesta es la acción que acaba de jugar.
\end{frame}

\begin{frame}
    \frametitle{Demostración del principio de estabilidad}
    \begin{lemma}[1]
        Dada una secuencia de juego ficticio $(i^\tau, j^\tau)_{\tau \in \mathbb{N}}$ con secuencias de creencias $x^\tau$ y $y^\tau$. Si $(i^k, j^k)$ es un equilibrio de Nash, entonces $i^k \in BR_1(y^{k})$ y $j^k \in BR_2(x^{k+1})$.
    \end{lemma}
\end{frame}

\begin{frame}
    \frametitle{Demostración del principio de estabilidad}
    \begin{align*}
        BR_1(y^k) &= BR_1(\frac{k - 1}{k} y^{k-1} + \frac{ \pstrat{j^k}}{k}) \\
        &= \argmax_{i \in N} \{\pstrat{i} A(\frac{(k - 1)y^{k-1}}{k} + \frac{ \pstrat{j^k}}{k})\}\\
        &= \argmax_{i \in N} \{\frac{(k - 1)}{k}\pstrat{i} Ay^{k-1} + \frac{1}{k}\pstrat{i} A \pstrat{j^*}\}
    \end{align*}
    \begin{itemize}
        \item \pause The definition of Fictitious Play implies that $i^k \in BR_1(y^{k-1}) = \argmax_{i \in N}\{\pstrat{i}Ay^{k-1}\}$.
        \item \pause The definition of Nash Equilibrium implies that $i^k \in BR_1(\pstrat{j^k}) = \argmax_{i \in N}\{\pstrat{i} A \pstrat{j^k}\}$.
        \item \pause $\frac{(k - 1)}{k}\pstrat{i}Ay^{k-1} + \frac{1}{k}\pstrat{i} A \pstrat{j^k}$ is just a linear combination.
        \item \pause Analogous for player 2.
    \end{itemize}
\end{frame}

\begin{frame}
    \frametitle{Demostración del principio de estabilidad}
    \begin{lemma}[2]
        Given a Fictitious Play sequence $(i^\tau, j^\tau)_{\tau \in \mathbb{N}}$ with belief sequences $x^\tau$ and $y^\tau$. If $(i^k, j^k)$ is a Nash Equilibrium then $BR_1(y^{k}) \subseteq BR_1(y^{k-1})$ and $BR_2(x^{k+1}) \subseteq BR_2(x^{k})$.
    \end{lemma}
    \pause If the process plays a Nash equilibrium then there are no new actions in the best response sets in the next iteration.
\end{frame}

\begin{frame}
    \frametitle{Demostración del principio de estabilidad}
    \begin{itemize}
        \item We can prove $BR_1(y^{k}) \subseteq BR_1(y^{k-1})$ by proving $i \notin BR_1(y^{k-1}) \implies i \notin BR_1(y^{k})$.
        \item \pause If $i^k \in BR_1(y^{k-1})$ but $i' \notin BR_1(y^{k-1})$ then $\pstrat{i^k}Ay^{k-1} > \pstrat{i'}Ay^{k-1}$
        \item \pause $(i^k, j^k)$ is a Nash Equilibrium so $\pstrat{i^k}A\pstrat{j^k} \ge \pstrat{i'}A\pstrat{j^k}$.
    \end{itemize}
    \pause
    \begin{align*}
        \pstrat{i^k}Ay^k &= \pstrat{i^k}A(\frac{k - 1}{k} y^{k-1} + \frac{\pstrat{j^k}}{k}) = \frac{(k - 1)}{k} \pstrat{i^k}Ay^{k-1} + \frac{1}{k}\pstrat{i^k}A\pstrat{j^k}\\
        &> \frac{(k - 1)}{k} \pstrat{i'}Ay^{k-1} + \frac{1}{k}\pstrat{i'}A\pstrat{j^k} = \pstrat{i'}A(\frac{k - 1}{k} y^{k-1} + \frac{\pstrat{j^k}}{k}) = \pstrat{i'}Ay^k
    \end{align*}
    \pause If the expected payoff for $i^k$ is greater than for $i'$ and $i^k$ is in the best response set, then $i'$ is not.
    
    \begin{quote}[Player 2]
        omg, same!
    \end{quote}
\end{frame}

\begin{frame}
    \frametitle{Rango de imagen}
        \begin{itemize}
        \item 
        \end{itemize}
\end{frame}

\begin{frame}
    \frametitle{Lema de preservación}
        \begin{itemize}
        \item 
        \end{itemize}
\end{frame}

\begin{frame}
    \frametitle{Demostración del lema de preservación}
        \begin{itemize}
        \item 
        \end{itemize}
\end{frame}

\begin{frame}
    \frametitle{Vel. de convergencia de AFP en simétricos}
        \begin{itemize}
        \item 
        \end{itemize}
\end{frame}

\begin{frame}
    \frametitle{Vel. de convergencia de AFP en intereses idénticos}
        \begin{itemize}
        \item 
        \end{itemize}
\end{frame}

\begin{frame}
    \frametitle{Conclusiones}
        \begin{itemize}
        \item Se continuó la línea de Brandt, Fischer y Harrenstein para mostrar que AFP puede requerir una cantidad exponencial de rondas antes de alcanzar un NE, evidenciando que su utilidad es criticable para este fin.
        \item Se han encontrado juegos minimales (no degenerados) que dan lugar a secuencias de AFP de longitud exponencial sin alcanzar un NE puro.
        \item Más aún, estos juegos son extensibles a tamaños mayores arbitrarios manteniendo las condiciones referidas.
        \item Hay casos en que aplicar SFP o AFP puede hacer diferencia en lo visto.
        \end{itemize}
\end{frame}

\begin{frame}
    \frametitle{Trabajo futuro}
        \begin{itemize}
        \item Lema de preservación para otras familias de juegos (y ver en cuáles no vale).
        \item Caracterizar mejor en qué juegos hay diferencias entre las velocidades de convergencia de SFP y AFP. 
        \item Combinaciones de AFP con otros algoritmos para encontrar equilibrios de Nash.
        \item Considerar $BR_1(y^{\tau-k_1})$, $BR_2(x^{\tau-k_2})$ en la ronda $\tau$.
        \item Juegos de más de 2 jugadores.
        \end{itemize}
\end{frame}

\end{document}
