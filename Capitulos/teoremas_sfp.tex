\begin{theorem}
    En juegos de suma constante simétricos de dos jugadores, SFP puede requerir una cantidad exponencial de rondas
    (en el tamaño de representación del juego) antes de que un equilibrio sea jugado.
\end{theorem}

\begin{proof}
    Consideremos un juego con matriz de pagos como la de la tabla \ref{tabla:teo1} con $\epsilon < 1$. Vemos que $(a_3, a_3)$ es el único
    equilibrio de Nash por ser el único perfil restante luego de realizar eliminación iterada de estados estrictamente dominados.

    \begin{center}
    \begin{tabular}{llll}
    .                          & $a^1$      & $a^2$      & $a^3$                            \\ \cline{2-4}
    \multicolumn{1}{l|}{$a^1$} & 0          & -1         & \multicolumn{1}{l|}{$-\epsilon$} \\
    \multicolumn{1}{l|}{$a^2$} & 1          & 0          & \multicolumn{1}{l|}{$-\epsilon$} \\
    \multicolumn{1}{l|}{$a^3$} & $\epsilon$ & $\epsilon$ & \multicolumn{1}{l|}{0}           \\ \cline{2-4}
    \end{tabular}
\end{center}

    Consideremos un número $k > 1$ arbitrario. Mostraremos que para $\epsilon = 2^(-k)$, SFP puede tomar $2^k$ rondas antes de que
    alguno de los jugares juegue $a_3$. Como el juego puede codificarse en $O(k)$ bits, esto demuestra el teorema.

    Si el proceso de SFP comienza con $(a_1, a_1)$, entonces $Py^i = x^iQ = (0, 1, 2^{-k})$ y por lo tanto ambos jugadores jugaran $a_2$
    en la segunda ronda.

    Luego, continuarán jugando $a_2$ hasta la ronda $2^k$ ya que para los $i$ tales que $1 \le i \le 2^k$, tendremos $Py^i = (-i + 1, 1, 2^{-k}i)$


\end{proof}