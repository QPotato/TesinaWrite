\chapter{Introducción} \label{cap:intro}

% Teoría de Juegos
Creada por John Von Neumann y Oskar Morgenstern, la teoría de juegos surgió en la década de 1940 motivada en parte por la Segunda Guerra Mundial. Estudia una variedad de situaciones en las que diversos agentes interactúan y toman decisiones estratégicas con el fin de obtener beneficios con interdependencias. Para ello, la teoría estudia los comportamientos, la posibilidad de maximizar esas ganancias según variados criterios y los casos de equilibrio, a la vez que formula y analiza modelos.

% Aplicaciones a la Economía de la Teoría de Juegos

% Teoría de Juegos Algorítmica y su lugar en las Ciencias de la Computación

% Equilibrios de Nash. Intuición de FP como explicación de EdN

% Historización de FP
  % Algo de AFP

% Investigación de la convergencia

% Investigación de la velocidad de convergencia
  % Algo de las diferentes métricas

% Algo de que AFP sigue estando sub-explorado

\section{Objetivos} \label{cap:intro:sec:obj}

Intro \falta{estoy en proceso de revisar la propuesta y las intros de los papers para armar esta}

