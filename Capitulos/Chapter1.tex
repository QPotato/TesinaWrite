\chapter{Titulo del capitulo}  \label{cap1}

\section{Titulo de la seccion} \label{cap1:sec1}

\begin{theorem}
    En juegos de suma constante simétricos de dos jugadores, SFP puede requerir una cantidad exponencial de rondas
    (en el tamaño de representación del juego) antes de que un equilibrio sea jugado.
\end{theorem}

\begin{proof}
    Consideremos un juego con la siguiente matriz de pagos:

    \begin{center}
    \begin{tabular}{llll}
    .                         & $a^1$       & $a^2$       & $a^3$                             \\ \cline{2-4}
    \multicolumn{1}{l|}{$a^1$} & 0          & -1         & \multicolumn{1}{l|}{$-\epsilon$} \\
    \multicolumn{1}{l|}{$a^2$} & 1          & 0          & \multicolumn{1}{l|}{$-\epsilon$} \\
    \multicolumn{1}{l|}{$a^3$} & $\epsilon$ & $\epsilon$ & \multicolumn{1}{l|}{0}           \\ \cline{2-4}
    \end{tabular}
\end{center}

    Si $\epsilon < 1$, vemos que $(a^3, a^3)$ es el único
    equilibrio de Nash por ser el único perfil restante luego de realizar eliminación iterada de estados estrictamente dominados.

    Consideremos un número $k > 1$ arbitrario. Mostraremos que para $\epsilon = 2^{-k}$, SFP puede tomar $2^k$ rondas antes de que
    alguno de los jugadores juegue $a^3$. Como el juego puede codificarse en $O(k)$ bits, esto demuestra el teorema.

    Si el proceso de SFP comienza con $(a^1, a^1)$, entonces $Py^1 = x^1Q = (0, 1, 2^{-k})$ y por lo tanto ambos jugadores jugaran $a_2$
    en la segunda ronda.

    Luego, continuarán jugando $a^2$ hasta la ronda $2^k$ ya que para los $i$ tales que $1 \le i \le 2^k$,
    tendremos $Py^i = (-i + 1, 1, 2^{-k}i)$. Mientras valga $2^{-k}i < 1$, ambos jugadores elegirán $a^2$ en la ronda $i + 1$

    Por lo tanto, la secuencia

    \begin{math}
        (a^1, a^1), (a^2, a^2), ... (a^2, a^2)
    \end{math}

    es una secuencia de aprendizaje válida de este juego que es exponencialmente larga en $k$ y en al cúal no se juega ningún equilibrio.
\end{proof}

\begin{theorem}
    En juegos de suma constante simétricos de dos jugadores, AFP puede requerir una cantidad exponencial de rondas
    (en el tamaño de representación del juego) antes de que un equilibrio sea jugado.
\end{theorem}

\begin{proof}
    Consideremos un juego con la siguiente matriz de pagos:

    \begin{center}
    \begin{tabular}{llll}
    .                         & $a^1$       & $a^2$       & $a^3$                             \\ \cline{2-4}
    \multicolumn{1}{l|}{$a^1$} & 0          & -1         & \multicolumn{1}{l|}{$-\epsilon$} \\
    \multicolumn{1}{l|}{$a^2$} & 1          & 0          & \multicolumn{1}{l|}{$-\epsilon$} \\
    \multicolumn{1}{l|}{$a^3$} & $\epsilon$ & $\epsilon$ & \multicolumn{1}{l|}{0}           \\ \cline{2-4}
    \end{tabular}
\end{center}

    Si $\epsilon < 1$, vemos que $(a^3, a^3)$ es el único
    equilibrio de Nash por ser el único perfil restante luego de realizar eliminación iterada de estados estrictamente dominados.

    Consideremos un número $k > 1$ arbitrario. Mostraremos que para $\epsilon = 2^{-k}$, AFP puede tomar $2^k$ rondas antes de que
    se juegue $(a^3, a^3)$. Como el juego puede codificarse en $O(k)$ bits, esto demuestra el teorema.

    Si el proceso de AFP comienza con el jugador fia jugando $a^1$, entonces $x^1Q = (0, 1, 2^{-k})$ y por lo tanto
    el jugador columna elegirá $a_2$.

    Luego, en las siguentes $2^k - 1$ rondas tendremos que $Py^i = (-i + 1, 0, 2^{-k}i)$ y $x^{i+1}Q = (-2^{-k}i, 1 - 2^{-k}i, 2^{-k}i$.
    Claramente, el jugador fila elegirá $a^3$. Por su parte, como $1 - 2^{-k}i \ge 2^{-k}$ el jugador columna jugará $a^2$ en la ronda $i + 1$.

    Por lo tanto, la secuencia

    \begin{math}
        (a^1, a^2), \underbrace{(a^3, a^2), ... (a^3, a^2)}_{2^k - 1 veces}
    \end{math}

    es una secuencia de aprendizaje válida de este juego que es exponencialmente larga en $k$ y en al cúal no se juega ningún equilibrio.
\end{proof}