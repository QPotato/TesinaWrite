\chapter{Conclusiones}  \label{cap:conclusiones}

Hemos probado que el juego ficticio alternante puede requerir una cantidad exponencial de rondas antes de alcanzar un equilibrio de Nash puro, pero también puede converger en tiempos razonables para juegos en los que el simultaneo es exponencial. Para esto, hemos definido juegos minimales en tamaño que pueden ser extendidos manteniendo la misma propiedad.

\section{Trabajo Futuro} 

Hay mucho para profundizar en esta diferencia hallada sobre la velocidad de convergencia de las variantes de juego ficticio. Dado un juego que converge en un número exponencial de iteraciones en el juego ficticio simultaneo, ¿Qué propiedad debe cumplir para converger en un número constante o lineal de iteraciones en el alternante? Caracterizar mejor los juegos que cumplen esta propiedad puede avanzar la posibilidad de la utilización práctica del juego ficticio alternante en protocolos reales que requieren el alineamiento de incentivos de distintos sistemas.

Otra línea de trabajo involucra generalizar el juego ficticio considerando $BR(x^{\tau-k})$ con $k$ una constante, es decir, que los jugadores observen las decisiones varias rondas atrás.

Por último, interesan también otras nociones de convergencia y equilibrios mixtos, así como juegos de más jugadores.
