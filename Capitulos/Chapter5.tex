\chapter{Conclusiones y trabajo futuro}

La búsqueda de equilibrios de Nash es un problema central de la teoría de juegos y sus aplicaciones prácticas no son pocas. En particular, su equivalencia con la resolución de un problema de programación lineal lo hace muy interesante desde el punto de vista computacional.

En los 70 años que han pasado ya desde la proposición original de Brown, el juego ficticio ha sido extensivamente caracterizado en términos de su convergencia en distintas clases de juegos, así como la de muchas variantes que lo mejoran en algún aspecto. Una forma muy natural de pensar el algoritmo, propuesta por \cite{extending:pattern} pero que en sí se vuelve evidente al intentar escribir un programa que lo simule, es dividirlo en una etapa de toma decisiones y otra de actualizaciones de creencias. Las variantes del juego ficticio simultáneo cambian una o ambas de estas etapas, limitando o enriqueciendo las creencias de alguna forma o mejorando la heurística a algo más complejo que tomar sencillamente una mejor respuesta a la jugada media, pero siempre manteniendo esencialmente la delimitación entre ambas.

El juego ficticio alternante es distinto en este aspecto, porque rompe con esta estructura simultánea del algoritmo al mezclar estas etapas y darle ventaja a un jugador que siempre tendrá sus creencias una ronda más actualizadas que el otro. Pero al hacerlo, además resigna cierta elegancia matemática, ya que la toma simultánea e independiente de decisiones es un axioma fundacional de la teoría de juegos (precisamente de los juegos en forma normal). Es por esto que, incluso aunque presentara buenas propiedades de convergencia, mientras el interés fuera como explicación teórica de los equilibrios de Nash, el juego ficticio simultáneo puede de alguna manera resultar estéticamente más interesante.

La pregunta por la velocidad de convergencia del juego ficticio y su utilidad práctica como mecanismo eficiente de cálculo de equilibrios de Nash es realmente muy reciente, y las publicaciones al respecto se centran, como es de esperar, en la variante más estudiada. Pero no es inadecuado pensar, a priori, que la introducción de esta ventaja a favor del segundo jugador que representa agregarle adelantadamente información que su rival no tuvo, puede significar computacionalmente la mejora de la eficiencia si de lo que se trata es que el mecanismo de aprendizaje se acerque al equilibrio tan rápido como se pueda.

En este trabajo, motivados por el estudio realizado por Brandt, Fischer y Harrenstein en \cite{brandt:rate:convergence}, hemos encontrado un primer ejemplo donde esta sospecha se confirma y este cambio en el algoritmo efectivamente resulta en una reducción de la complejidad algorítmica respecto a la variante simultánea. También hemos encontrado casos para los cuales la variante alternante no presenta una mejora en este aspecto y tiene las mismas cotas superiores exponenciales.

En \cite{brandt:rate:convergence} se menciona la existencia del juego ficticio alternante y se comenta brevemente que sus resultados pueden ser “extendidos fácilmente” al mismo. Si bien pudimos demostrar fácilmente el teorema \ref{teorema:afp:velocidad:simetricos} haciendo un desarrollo similar al de ellos pero con la actualización alternante de creencias, el teorema \ref{teorema:afp:velocidad:nondegen} requirió diseñar un juego con los valores justos para que la secuencia se extienda exponencialmente. Como se ve luego en el teorema \ref{teorema:afp:mejor}, la diferencia de incentivos entre las jugadas era ya suficiente en el juego ficticio alternante para que la convergencia sea inmediata. La idea fue entonces, plantear una variación en la que esta diferencia sea marginal y dependa de la variable $k$ y de la dimensión de las matrices.

Estos resultados son sólo un primer paso y puede profundizarse más sobre esta diferencia hallada. Sería deseable una caracterización más detallada sobre qué juegos presentan esta diferencia entre las velocidades de convergencia del juego ficticio simultáneo y el alternante. Pero, además, muchas de las variantes de juego ficticio simultáneo para las cuales se ha investigado la velocidad de convergencia pueden plantearse también de forma alternante y sus estudios extendidos de forma similar a como hemos hecho aquí. Lo mismo ocurre con los métodos combinados que mezclan juego ficticio con otros algoritmos tales como las mecánicas de no arrepentimiento o el método Simplex sobre el problema de programación lineal equivalente.

Otro eje a tener en cuenta es que nos hemos limitado al estudio de la convergencia de secuencias de creencias a un equilibrio de Nash puro. Y puede resultar interesante estudiar también la velocidad de convergencia a un equilibrio mixto y si esta cambia según la variante de juego ficticio considerada. Además, el juego ficticio simultáneo puede definirse también para juegos de más de dos jugadores, como se hace en \cite{no:cycling}, aunque no es evidente cómo conviene extender esta definición, de ser posible, a la variante alternante. 