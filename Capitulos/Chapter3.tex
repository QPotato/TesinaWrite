\chapter{Conceptos Previos}  \label{cap1}

\section{Teoría de Juegos}
[TODO: Juegos en forma normal]
[TODO: Juegos en forma bimatricial]

\section{Juego Ficticio}
Presentaremos ahora, la definición de Juego Ficticio Simultaneo (SFP) que usan Berger, Shapley, Monderer y Sela
\cite{browns:original} \cite{strategic:complementarities} [TODO: citar shapley:counter] \cite{no:cycling}, \cite{identical:interests}.


\begin{definition}
    Sea $(A, B)$ un juego en forma bimatricial de $n \times m$.
\end{definition}

Alternativamente, Brandt, Fischer y Harrenstein utilizan una definición equivalente que resulta más comoda para estudiar velocidades
de convergencia:

\begin{definition}
    Sea $(A, B)$ un juego en forma bimatricial de $n \times m$.
\end{definition}

Esta definición es a su vez muy similar a la que utiliza Robinson \cite{robinson:zerosum}
\section{Propiedad del Juego Ficticio}