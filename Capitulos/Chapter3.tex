\chapter{Conceptos Previos}  \label{cap:previos}

\section{Juegos Bimatriciales y Equilibrios de Nash}

% Comenzaremos dando una definición de juego estratégico adaptada de \cite{libro:rubinstein}:
% % Tomada de Osbourne & Rubinstein
% \begin{definition}
%     Un \concept{juego estrategico} consiste en
%     \begin{itemize}
%         \item Un conjunto finito $N$ (el conjunto de \concept{jugdores}).
%         \item Para cada jugador $i \in N$, un conjunto no vacío $A_i$ (el conjunto de \concept{acciones} disponibles). Notamos con $A$ al conjunto $ $ y llamamos \concept{perfil} de acciones a sus elementos.
%         % TODO: producto cartesiano en todos los indices
%         \item Para cada jugador $i \in N$, una \concept{función de pagos} $u_i : A \rightarrow \R$ que a cada perfil de acciones $a$ le asigna un valor real que llamaremos la \concept{utilidad} o \concept{pago} del perfil $a$ para el jugador $i$.
%     \end{itemize}
% \end{definition}

% En este trabajo, nos enfocaremos en los juegos estratégicos de dos jugadores. Suponiendo $N = \{1, 2\}$ Estos juegos pueden expresarse de manera compacta en \concept{forma normal} mediante un par de matrices $(A, B)$ tales que $A_{ij} = u_1(a_i)

% Definicion de bimatricial tomada de brown's original
Sea $(A, B)$ un juego en forma bimatricial de $n \times m$, es decir un juego de dos jugadores finito en el que el jugador $1$ (jugador fila) tiene acciones $i \in N = \{1, 2, ..., n\}$ y el jugador $2$ (jugador columna) tiene acciones $j \in M = \{1, 2, ..., m\}$. $A$, $B \in \mathbb{R}^{n \times m}$ son las matrices de pago de los jugadores $1$ y $2$. Si el jugador $1$ elige la acción $i$ y el jugador $2$ elige la acción $2$, la utilidad del jugador $1$ será $a_{ij}$ y la utilidad del jugador $2$ será $b_{ij}$.

% Estrategias mixtas como vectores en conjunto delta
Notaremos con $\Delta(X)$ al espacio de probabilidades sobre el conjunto $X$ y llamaremos estrategias mixtas a los vectores de la forma $x \in \Delta(N)$ y $y \in \Delta(M)$. A las estrategias que asignan probabilidad $1$ a una acción y $0$ a todas las otras, las llamaremos estrategias puras. Notaremos con $\pstrat{h}$ \cambios{metí esta notación para estrategias puras, para no hacer el abuso de notacion (y sacarme de encima esos $u_i$ y $v_i$ de Brandt)} a la estrategia pura correspondiente a la acción $h$.

% Expected payoff como producto de matriz por vector
La utilidad esperada del jugador $1$ al jugar la acción $i$ contra la estrategia mixta $y$ del jugador $2$ será $(Ay)_i$. Analogamente, la utilidad esperada del jugador $2$ al jugar la acción $j$ contra la estrategia mixta $x$ del jugador $2$ será $(xB)_j$ \cambios{aca cambié a $xB$ para no tener que estar haciendo transposiciones}. Si ambos jugadores juegan estrategias mixtas $x$ e $y$ respectivamente, sus utilidades esperadas podrán calcularse como $xAy$ para el jugador $1$ y $yB^tx$ para el jugador $2$.

% Best-response
Si $y$ es una estrategia mixta del jugador $2$, definimos el conjunto de mejores respuestas a $y$ como $BR_1(y) = \{i \in N : (Ay)_i = \max_{i' \in N} (Ay)_i'\}$ y análogamente, si $x$ es una estrategia mixta del jugador $2$, el conjunto de mejores respuestas a $x$ será $BR_2(x) = \{j \in N : (xB)_j = \max_{j' \in M} (xB)_{j'}\}$. \cambios{Arreglé un poquito esta def}

Llamaremos equilibrio de Nash a todo perfil de estrategias mixtas $(x*, y*) \in \Delta(A) \times \Delta(B)$ que cumpla:
\begin{gather}
    \forall i \in N, x^*_i > 0 \implies i \in BR_1(y^*) \\
    \forall j \in M, y^*_j > 0 \implies j \in BR_1(x^*)
\end{gather}

En el caso particular de que las estrategias sean puras, al equilibrio lo llamaremos también equilibrio de Nash puro.

\falta{aca tiene que ir algunos ejemplos de juegos conocidos}

\section{Categorías de Juegos}

Es útil clasificar a los juegos en distintas categorías según sus propiedades. Presentamos a continuación algunas de las categorías de juegos mas estudiadas en la literatura sobre juego ficticio.

\begin{definition}
    Un juego $(A, B)$ de tamaño $n \times m$ es un juego de suma cero si $\forall i \in N, j \in M : a_{ij} = -b_{ij}$. Para estos casos, nos referiremos al juego usando solo la matriz $A$.
\end{definition}

Los juegos de suma cero son una de las categorías mas estudiadas en teoría de juegos. Representan los juegos en los que un jugador siempre gana tanto como pierde el otro. Uno de los teoremas fundacionales del area, conocido como el Teorema de Minimax \cite{nash:minmax} establece que todos poseen un equilibrio de Nash puro y a la utilidad para el jugador fila de este perfil la llama el \concept{valor del juego}.

\begin{definition}
    Llamamos degenerado a un juego bimatricial $(A, B)$ de tamaño $n \times m$ si se cumple alguna de las siguientes condiciones:
    \begin{itemize}
        \item Existen $i, i' \in N$ y $j \in M$ con $i \neq i$' tales que $a_{ij} = a_{i'j}$
        \item Existen $j, j' \in M$ e $i \in N$ con $j \neq j$' tales que $b_{ij} = b_{ij'}$
    \end{itemize}
\end{definition}
Los juegos no degenerados son de particular interes porque capturan el concepto de un juego en el que, para cada acción del rival, no existen dos acciones con el mismo pago. Por lo tanto, el conjunto de mejor respuesta contra una acción dada es siempre unitario.

\falta{comentarios de estas dos clases y por que las definimos}
\begin{definition}
    Un juego bimatricial $(A, B)$ de tamaño $n \times n$ es simétrico si $\forall i, j \in N : a_{ij} = -b_{ji}$
\end{definition}

\begin{definition}
    Un juego $(A, B)$ de tamaño $n \times m$ es de intereses idénticos si $\forall i \in N, j \in M : a_{ij} = b_{ij}$.
\end{definition}

\falta{aca pueden ir comentarios sobre cuales de los juegos ejemplo caen en cuales categorías}

\section{Juego Ficticio} \label{sec:def:fp}

\todo[color=blue, inline]{Mas intro acá. Explicación coloquial. Capaz algo de la division del algoritmo en decision y actualizacion como dice en pattern:matching}

Existen en la literatura dos formas de definir el juego ficticio. Algunos autores como Berger, Monderer, Sela, Shapley, Daskalakis y Pan \cite{browns:original} \cite{no:cycling} \cite{2x2:without} \cite{identical:interests} \cite{counter:karlin:strong} utilizan un definición del estilo de la siguiente, que resulta cómoda para estudiar convergencia y es la que veremos primero. Presentaremos dos variantes, correspodientes a la versión simultanea y a la alternante.


\begin{definition} \label{def:fp:berger}
    Sean $(A, B)$ un juego en forma bimatricial de $n \times m$ y una secuencia $(i^\tau, j^\tau)$ con $i^\tau \in N$, $j^\tau \in M$ para todo $\tau \in \mathbb{N}$. Si tenemos unas secuencias de creencias $x^\tau$ e $y^\tau$ tales que para todo $\tau \in \mathbb{N}$: \cambios{con notacion nueva y especificando que los elementos son acciones}
    \begin{gather*}
        x^\tau= \frac{\sum^\tau_{s=1} \pstrat{i^s}}{t}  \\
        y^\tau= \frac{\sum^\tau_{s=1} \pstrat{j^s}}{t}
    \end{gather*}
    Entonces:
    \begin{itemize}
        \item $(i^\tau, j^\tau)$ es una secuencia de juego ficticio simultaneo si $(i^1, j^1)$ es un elemento arbitrario de $N \times M$ y para todo $\tau \in \mathbb{N}$ se cumplen $i^{\tau+1} \in BR_1(y^\tau)$ y $j^{\tau+1} \in BR_2(x^\tau)$.
        \item $(i^\tau, j^\tau)$ es una secuencia de juego ficticio alternante si $i^1$ es un elemento arbitrario de $N$ y para todo $\tau \in \mathbb{N}$ se cumplen $i^{\tau+1} \in BR_1(y^\tau)$ y $j^{\tau} \in BR_2(x^\tau)$.
    \end{itemize}
\end{definition}

Vale la pena hacer algunos comentarios sobre esta primer definición. Para empezar, si observamos las secuencias de creencias, $x^\tau$ e $y\tau$, veremos que se componen de sumas de vectores unitarios, pero normalizados por el tiempo, por lo que son estrategias mixtas que se corresponden con la distribución empírica de las acciones de cada jugador.

Por otro lado, como los elementos iniciales son arbitrarios, un mismo juego puede tener tantos procesos de juego ficticio validos como jugadas iniciales existan. Mas aún, existe una ambigüedad en cuanto que acciones elige el proceso en los casos en los que uno o ambos de los conjuntos de mejor respuesta no son unitarios\footnote{Es por esto que es tan imporante en el estudio del juego ficticio la categoría de los juegos no degenerados, para los cuáles el conjunto de mejor respuesta es unitario.}. Cada posible decisión configura un proceso distinto y válido pero es normal hablar de \concept{reglas de desempate} para referirse a criterios uniformes para resolver estas ambigüedades. Más formalmente, podemos definir una regla de desempate como un par de funciones $d_1: \mathbb{P}(N) \rightarrow N$ y $d_2: \mathbb{P}(M) \rightarrow M$ que a cada subconjunto de acciones de un jugador en un eventual empate le asignan la acción que elegira el proceso. \cambios{inicial y reglas de desempate}

Sobre la variante alternante vale aclarar que efectivamente, el jugador columna toma su decisión incorporando en sus creencias la información sobre qué jugó el jugador fila en la ronda actual, si bien esto puede resultar poco intuitivo ya que normalmente en teoría de juegos se representa jugadores eligiendo simutanea e independientemente. Otra observación interesante es que su acción inicial no es arbitraria sino que ya se encuentra fijada por la mejor respuesta o mejores respuestas a la acción del jugador fila. \cambios{aclaración de alternante}

Diremos que un proceso de juego ficticio (simultaneo o alternante) converge de forma pura en la iteración $k$ si $(i^k, j^k)$ es un equilibrio de Nash puro. En el capítulo \ref{cap:aportes} veremos que cuando esto ocurre, $(i^k, j^k)$ se repetirá infinitamente desde este punto en el tiempo. Diremos también que el proceso converge de forma mixta si existe un equilibrio de Nash mixto tal que para todo $\epsilon > 0$ existe un $k \in \mathbb{N}$ tal que $|x^* - x^k| < \epsilon$ y $|y^* - y^k| < \epsilon$ \cambios{defs de convergencia}. \todo[inline]{esto podría ir en una sección propia y de paso meter ahi representacion de bits y convergencia en funcion de que}

\falta{aca puede ir un desarrollo de un FP para simultaneo y alternante con algun juego de los ejemplos}

Alternativamente, Brandt, Fischer y Harrenstein utilizan una definición similar a la de Robinson \cite{robinson:zerosum} pero simplificada. Es más cómoda para estudiar velocidades de convergencia en juegos que se sabe que convergen.

\begin{definition} \label{def:fp:brandt}
    Sea $(A, B)$ un juego en forma bimatricial de $n \times m$:
    \begin{itemize}
        \item Una secuencia de juego ficticio simultaneo en $(A, B)$ es una secuencia $(x^0, y^0), (x^1, y^1), (x^2, y^2), \dots$ de pares de vectores no negativos $(x^i, y^i) \in \mathbb{N}^n \times \mathbb{N}^m$ tal que:\cambios{con notacion nueva}
        \begin{gather*}
            x^0 = \boldmath{0}, y^0 = \boldmath{0} \\
            x^{\tau+1} = x^{\tau} + \pstrat{i} \text{\ donde $i$ es el índice de una componente máxima de $Ay^\tau$} \\
            y^{\tau+1} = x^{\tau} + \pstrat{j} \text{\ donde $j$ es el índice de una componente máxima de $x^{\tau}B$}
        \end{gather*}
        \item Una secuencia de juego ficticio alternante en $(A, B)$ es una secuencia $(x^0, y^0), (x^1, y^1), (x^2, y^2), \dots$ de pares de vectores no negativos $(x^i, y^i) \in \mathbb{N}^n \times \mathbb{N}^m$ tal que:
        \begin{gather*}
            x^0 = \boldmath{0}, y^0 = \boldmath{0} \\
            x^{\tau+1} = x^{\tau} + \pstrat{i} \text{\ donde $i$ es el índice de una componente máxima de $Ay^\tau$} \\
            y^{\tau+1} = y^{\tau} + \pstrat{j} \text{\ donde $j$ es el índice de una componente máxima de $x^{\tau+1}B$}
        \end{gather*}
    \end{itemize}
\end{definition}

Como podemos observar, la principal diferencia con la primer definición es que mientras en aquella se define una secuencia de jugadas que cumple una condicion contra un historial de creencias sobre la estrategia mixta del otro jugador, en esta la secuencia es de duplas de contadores de jugadas (sin normalizar, por lo que no son estrategias mixtas), que cumplen en cada iteración una condición contra producto de un elemento de la otra contra las matrices de pago, pero sin ser este tampoco el valor esperado de la jugada. En el capítulo \ref{cap:aportes} se demostrará que estas dos definiciones son equivalentes. Como en esta definición aparecen también casos ambigüos y aplica la misma noción de reglas de desempate. \cambios{este lo reescribí un poco también}

\falta{aca puede ir el mismo ejemplo de la def anterior, pero con esta}

% Convergencia en pagos

% Convergencia en creencias

% Relacion entre distintas convergencias

