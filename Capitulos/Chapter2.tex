\chapter{Estado del Arte}  \label{cap:relwork}

\section{Convergencia del Juego Ficticio}

% propuesta original y convergencia zero sum
El proceso de aprendizaje de juego ficticio fue propuesto por primera vez por Brown en 1951 \cite{brown:1951} como un algoritmo para encontrar el valor de un juego de suma zero finito. Hacia finales del mismo año, Robinson \cite{robinson:zerosum} demostró que el proceso converge para todos los juegos de esta clase.

% convergencia de SFP por tamaños
Desde entonces, se han publicado numerosos trabajos analizando la convergencia del juego ficticio en juegos que no sean de suma zero. Miyazawa \cite{miyazawa:2x2} demostró que esta propiedad vale para todos los juegos de $2 \times 2$ pero, su demostración depende de la incorporación de una regla de desempate particular sin la cual, Monderer y Sela \cite{2x2:without} demostraron que no se cumple. Por su parte, Shapley \cite{shapley:3x3} mostró un ejemplo de un juego de $3 \times 3$ para el cuál no es válida.

% convergencia de SFP por clases
Además, la convergencia del juego ficticio fue demostrada para juegos de intereses identicos \cite{identical:interests}, juegos potenciales con pesos \cite{weighted:potential}, juegos no degenerados con estrategias complementarias y ganancias disminuyentes \cite{strategic:complementarities} y ciertas clases de juegos compuestos \cite{compound}.

% estudios de variantes
% TODO: ver si detallar las ultimas
Por otro lado, se han estudiado muchas variantes del juego ficticio. Una de las mas analisadas es el juego ficticio continuo, definida originalmente en la publicación original de Brown \cite{brown:1951}, aunque este no la exploró en detalle. Monderer y Sela \cite{no:cycling} demostraron que esta converge para juegos no degenerados de $2 \times 3$ y Berger luego extendio este resultado a $2 \times N$ \cite{berger:2xn} y aportó la convergencia de los juegos de potencial ordinal y quasi-supermodulares con ganancias disminuyentes. Otros ejemplos de variantes propuestas peuden verse en \cite{pattern:recog} y \cite{new:kind:fp}.

% AFP
% TODO: exapndir
Una variante de particular interes este trabajo es el juego ficticio con actualización alternante de creencias. Berger \cite{browns:original} planteó que esta versión alternante es en realidad la original que definió Brown en \cite{browns:original} y que si bien el proceso con actualización simultanea de creencias que usan todos los investigadores de teoría de juegos en la actualidad puede resultar mas intuitivo, es también menos potente y da como ejemplo la clase de los juegos no degenerados con potencial ordinal para la cuál la version alternante converge, pero la simultanea no.

\section{Velocidad de Convergencia del Juego Ficticio}

% Relaciones con otros procesos/teorias/metodos
% velocidad de convergencia
Los trabajos mencionados hasta ahora se enfocan en el estudio de la eventual convergencia global a un equilibrio de Nash de las distintos clases de juegos. Otra enfoque de investigación es la velocidad de convergencia en los casos en la que esta ocurre. El interés por este se debe en gran medida a la equivalencia entre los juegos de suma cero y los problemas de programación lineal, demostrada por Dantzig, Gale y Von Neumann \cite{fplp:equiv} \cite{programming:game:equivalence}. En 1994, Gass y Zafra \cite{modified:fp:linear} planteaban que hasta la fecha, lo más eficiente para resolver un juego de suma cero era plantearlo como un problema de programación lineal y aplicar el método simplex. En el mismo artículo plantean un meotodo mixto con simplex y una variante de juego ficticio y concluyen que permite acelerar la convergencia en ciertos problemas de programación lineal. Lambert y Smith \cite{aproach:large:scale} plantean también una variante (con muestreo) y discuten su eficiencia en problemas de optimización a gran escala.

Vale la pena mencionar en este punto lo que en la literatura del tema se conoce como la Conjetura de Karlin. En 1959, Samuel Karlin \cite{karlin:conjecture} conjeturó que la velocidad de convergencia del juego ficticio es $O(t^{-\frac{1}{2}})$ para todos los juegos. La idea de proviene de que esta cota superior se corresponde con la de la velocidad de convergencia de otro método de aprendizaje muy relacionado con el juego ficticio, las dinámicas de no-arrepentimiento \cite{no:regret} \cite{no:regret:2}. Daskalakis y Pan \cite{counter:karlin:strong} probaron falsa una versión fuerte de la Conjetura de Karlin (usando una regla de desempate arbitraria) pero dejaron abierta la pregunta sobre la versión general, que ellos llaman débil, de la conjetura.

La utilidad del juego ficticio como método para computar equilibrios de Nash fue puesta en duda cuándo Brandt, Fischer y Harrenstein \cite{brandt:rate:convergence} demostraron que para los juegos de suma cero, los no degenerados de $2 \times N$ y los potenciales (tres de las clases mas estudiadas), existen casos en los que el proceso de juego ficticio puede tomar una cantidad de pasos exponecial en el tamaño de representación del juego antes de que se juegue algún equilibrio. En esta publicación mencionan que su resultado puede ser extendido al juego ficticio alternante pero no demuestran esto.

% TODO: casos de uso (poker IA segun bramdt, redes de trafico (en pattern matching?) y alguno mas)