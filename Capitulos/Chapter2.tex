\chapter{Estado del Arte}  \label{cap:relwork}

% \section{Convergencia del Juego Ficticio}

% propuesta original y convergencia zero sum
El proceso de aprendizaje de juego ficticio fue propuesto por primera vez por Brown en 1951 \cite{brown:1951} como un algoritmo para encontrar el valor de un juego de suma zero finito. Hacia finales del mismo año, Julia Robinson \cite{robinson:zerosum} demostró que el proceso converge a un equilibrio de nash puro para todos los juegos de esta clase, independientemente de las condiciones iniciales.

% convergencia de SFP
Desde entonces, se han publicado numerosos trabajos analizando la convergencia del juego ficticio en juegos que no sean de suma zero. Miyazawa \cite{miyazawa:2x2} demostró que esta propiedad vale para todos los juegos de $2 \times 2$ pero, su demostración depende de la incorporación de una regla de desempate particular. mientras que Shapley \cite{shapley:3x3} mostró un ejemplo de un juego de $3 \times 3$ para el cuál no es válida.

% estudios de variantes

% AFP

% Relaciones con otros procesos/teorias/metodos

% velocidad de convergencia

Los trabajos mencionados en al sección anterior se enfocan en el estudio de la eventual convergencia global
a un equilibrio de Nash de las distintos clases de juegos. Otra enfoque de investigación es la velocidad de
convergencia en los casos en la que esta ocurre.
El interés por este último se debe en gran medida a la equivalencia entre los juegos de suma cero y
los problemas de programación lineal, demostrada por Dantzig, Gale y Von Neumann \cite{fplp:equiv}.