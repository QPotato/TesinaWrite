\chapter{Resultados}  \label{cap:aportes}

\section{Equivalencia de las distintas definiciones}

Como mencionamos en la sección \ref{sec:def:fp}, existen dos formas de definir el juego ficticio entre los distintos autores de la literatura. Ambas simplifican de distintas formas la definición original de Brown y si bien son similares, su equivalencia no es inmediatamente evidente, por lo que uno podría dudar de si un teorema expresado para una de las definiciones es válido con la otra. Por lo tanto, presentamos a continuación dos lemas sobre esta equivalencia, para el caso simultáneo y alternante respectivamente. La idea será probar que los historiales de la definición \ref{def:fp:brandt} suman en cada iteración la estrategia pura correspondiente a la acción elegida por la definición \ref{def:fp:berger}. Comenzamos con el caso simultáneo.

\begin{lemma} \label{lema:equiv:sim}
    Sea $(A, B)$ un juego en forma bimatricial de $n \times m$, $(\pstrat{i^\tau}, \pstrat{j^\tau})_{\tau \in \mathbb{N}}$ una secuencia de juego ficticio simultáneo (según la definición \ref{def:fp:berger}) con secuencias de creencias $y^\tau$, $x^\tau$ y sea $(p^\tau, q^\tau)_{\tau \in \mathbb{N}}$ una secuencia de juego ficticio simultáneo (según la definición \ref{def:fp:brandt}), tales que $p^1 = \pstrat{i^1}$, $q^1 = \pstrat{j^1}$ y ambas usan las mismas reglas de desempate $(d_1, d_2)$. Entonces, $(i^\tau, j^\tau)_{\tau \in \mathbb{N}}$ y $(p^\tau, q^\tau)_{\tau \in \mathbb{N}}$ representan el mismo proceso de aprendizaje. Es decir, $\forall \tau \in \mathbb{N}$, se cumplen:

    \[ p^{\tau} = \sum_{s=1}^\tau{\pstrat{i^s}} \]
    \[ q^{\tau} = \sum_{s=1}^\tau{\pstrat{j^s}} \]

\end{lemma}
\begin{proof}
    Procederemos por inducción sobre $\tau$.

    Para $\tau = 1$, tenemos $p^1 = \pstrat{i^1} = \sum_{s=1}^1{\pstrat{i^s}}$ y $q^1 = \pstrat{j^1} = \sum_{s=1}^1{\pstrat{j^s}}$.

    % Este parrafo esta a medio reescrbir con las defs nuevas
    Veamos ahora el caso de un $\tau > 1$, suponiendo que $p^{\tau-1} = \sum_{s=1}^{\tau-1}{\pstrat{i^s}}$ y $q^{\tau-1} = \sum_{s=1}^{\tau-1}{\pstrat{j^s}}$.
    Por la definición \ref{def:fp:brandt}, $p^\tau = p^{\tau-1} + \pstrat{i}$
    donde $i = d_1(argmax_{i \in N}\{\pstrat{i}Ap^{\tau-1}\})$.
    Pero también sabemos, por la definición \ref{def:fp:berger}
    que $i^\tau = d_1(BR_1(y^{\tau-1})) = d_1(BR_1(\frac{\sum_{s=1}^{\tau-1} \pstrat{j^s}}{\tau-1})) = d_1(argmax_{i \in N}\{A\frac{\sum_{s=1}^{\tau-1} \pstrat{j^s}}{\tau-1})\}) = d_1(argmax_{i \in N}\{A\frac{q^{\tau-1}}{\tau-1})\})$.
    Como escalar un vector no afecta la relación de orden entre sus componentes,
    podemos afirmar también que $i^\tau = d_1(argmax_{i \in N}\{Aq^{\tau-1})\})$.
    Luego, aplicando esto a la definición \ref{def:fp:brandt}, $p^\tau = p^{\tau-1} + \pstrat{i^\tau} = \sum_{s=1}^{\tau-1}{\pstrat{i^s}} + \pstrat{i^\tau} = \sum_{s=1}^{\tau}{\pstrat{i^s}}$.
    
    Análogamente, $q^{\tau} = \sum_{s=1}^\tau{\pstrat{j^s}}$.
\end{proof}

En el caso alternante, la jugada del jugador columna ya no es análoga a la del jugador fila, y el análisis es un poco mas complejo.

\begin{lemma}
    Sea $(A, B)$ un juego en forma bimatricial de $n \times m$, $(i^\tau, j^\tau)_{\tau \in \mathbb{N}}$ una secuencia de juego ficticio alternante (según la definición \ref{def:fp:berger}) con secuencias de creencias $y^\tau$, $x^\tau$ y sea $(p^\tau, q^\tau)_{\tau \in \mathbb{N}}$ una secuencia de juego ficticio alternante (según la definición \ref{def:fp:brandt}), tales que $p^1 = \pstrat{i^1}$ y ambas usan las mismas reglas de desempate $(d_1, d_2)$. Entonces, $(i^\tau, j^\tau)_{\tau \in \mathbb{N}}$ y $(p^\tau, q^\tau)_{\tau \in \mathbb{N}}$ representan el mismo proceso de aprendizaje. Es decir, para todo $\tau \in \mathbb{N}$, se cumplen:

    \[ p^{\tau} = \sum_{s=1}^\tau{\pstrat{i^s}} \]
    \[ q^{\tau} = \sum_{s=1}^\tau{\pstrat{j^s}} \]

\end{lemma}
\begin{proof}
    Nuevamente, procederemos por inducción sobre $\tau$.

    Para $\tau = 1$, sabemos que $p^1 = \pstrat{i^1} = \sum_{s=1}^1{\pstrat{i^s}}$.
    Por la definición \ref{def:fp:brandt}, $q^1 = q^0 + \pstrat{j}$ donde $q^0$ es el vector nulo y
    $j = d_2(argmax_{j \in M}\{p^{1}B\pstrat{j}\}) = d_2(argmax_{j \in M}\{\pstrat{i^{1}}B\pstrat{j}\})$.
    Además, por la definición \ref{def:fp:berger}, sabemos que
    $j^1 = d_2(BR_2(x^1)) = d_2(BR_2(\frac{\sum^1_{s=1} \pstrat{i^s}}{\tau})) = d_2(BR_2(i^1)) = d_2(argmax_{j \in M}\{\pstrat{i^{1}}B\pstrat{j}\})$.
    Luego, $q^1 = \pstrat{j^1} = \sum_{s=1}^1{\pstrat{j^s}}$.

    Veamos ahora el caso de $\tau > 1$, suponiendo que
    $p^{\tau-1} = \sum_{s=1}^{\tau-1}{\pstrat{i^s}}$ y $q^{\tau-1} = \sum_{s=1}^{\tau-1}{\pstrat{j^s}}$.
    Podemos afirmar, con el mismo argumento que en el caso inductivo del lema \ref{lema:equiv:sim}, que
    $p^{\tau} = \sum_{s=1}^\tau{\pstrat{i^s}}$.
    Por otro lado, $q^{\tau} = y^{\tau} + \pstrat{j}$ donde
    $j = d_2(argmax_{j' \in M}\{p^{\tau+1}B\pstrat{j'}\}) = d_2(argmax_{j' \in M}\{\sum_{s=1}^\tau{\pstrat{i^s}}B\pstrat{j'}\})$.
    Sabemos además, por la definición \ref{def:fp:berger}, que
    $j^\tau = d_2(BR_2(x^{\tau})) = d_2(BR_2(\frac{\sum_{s=1}^{\tau}{\pstrat{i^s}}}{\tau})) = d_2(argmax_{j' \in N}\{\frac{\sum_{s=1}^{\tau}{\pstrat{i^s}}}{\tau}B\pstrat{j}\})  = d_2(argmax_{j' \in N}\{\frac{p^{\tau}}{\tau}B\pstrat{j}\})$.
    Como escalar un vector no afecta la relación de orden entre sus componentes, podemos afirmar también que
    $j^\tau = d_2(argmax_{j' \in N}\{p^{\tau}B\pstrat{j}\})$.
    Luego, aplicando esto a la definición \ref{def:fp:brandt},
    $q^\tau = q^{\tau-1} + j^\tau = \sum_{s=1}^{\tau-1}{\pstrat{j^s}} + j^\tau = \sum_{s=1}^{\tau}{\pstrat{j^s}}$.

\end{proof}


\section{Preservación del Juego Ficticio}

En esta sección estudiaremos cómo las transformaciones a un juego afectan las secuencias de juego ficticio sobre el mismo. Esto nos resulta de interés porque más adelante nos permitirá encarar con más claridad el estudio de la velocidad de convergencia del algoritmo. También, razonar sobre esto nos permite entender mejor los resultados sobre convergencia de la literatura.

Claramente, transformaciones que preserven las relaciones de orden de los pagos de las acciones en juegos del mismo tamaño (como escalar todos los pagos por un mismo factor o sumarle a todos una constante) preservarán las secuencias de juego ficticio.

Pero veamos qué pasa cuándo cambia el tamaño del juego. La motivación para esto es que, en general, los investigadores de teoría de juegos buscan dos tipos de resultados de convergencia sobre juego ficticio. Por la positiva, clases de juegos de tamaño lo ms grande posible (o incluso mejor, tamaño arbitrario) para los que toda secuencia converja. Por la negativa, contra-ejemplos de tamaño tan pequeño como sea posible que no converjan. Esto se debe a que en estos estudios se entiende implícitamente que un contra-ejemplo puede ser expandido a un juego de tamaño más grande que presente la misma secuencia de juego ficticio y por tanto todos los juegos de ese tamaño o mayor ya presentan casos que no convergen.

En los siguientes lemas, demostraremos esto para algunas de las clases de juegos más estudiadas en la literatura. Vale la pena aclarar que si bien podríamos demostrar un solo lema general que afirme que todo juego puede expandirse conservando las secuencias de juego ficticio, este resultado no nos sería útil, ya que es importante que la expansión conserve las propiedades que definen a la clase de juegos en cuestión.

\begin{lemma}
    Sea $A$ un juego de suma cero de tamaño $n$, existe un juego  de suma cero $A'$ de tamaño $n+1$ tal que toda secuencia $(i^\tau, j^\tau)_{\tau \in \mathbb{N}}$ de juego ficticio (secuencial o alternante) en $A$ lo es también en $A'$.
\end{lemma}
\begin{proof}
    Definimos el juego $A'$ como
    \begin{equation*}
        A' =
        \begin{pmatrix}
        a_{1,1} & a_{1,2} & \cdots & a_{1,n} & max_{j \in N} a_{1,j}    \\
        a_{2,1} & a_{2,2} & \cdots & a_{2,n} & max_{j \in N} a_{2,j}    \\
        \vdots  & \vdots  & \ddots & \vdots  & \vdots                   \\
        a_{n,1} & a_{n,2} & \cdots & a_{n,n} & max_{j \in N} a_{n,j}    \\
        min_{i \in N} a_{i,1} & min_{i \in N} a_{i,2} & \cdots & min_{i \in N} a_{i,n}        & 0
        \end{pmatrix}
    \end{equation*} \falta{cambiar esto a las matrices nuevas}

    Sea $(i^\tau, j^\tau)_{\tau \in \mathbb{N}}$ una secuencia de juego ficticio secuencial sobre A, con secuencias de creencias $(x^\tau, y^\tau)_{\tau \in \mathbb{N}}$. En la iteración $\tau$, sabemos que $i^\tau \in BR_1(y^{\tau - 1})$ en $A$. Como el pago de $n+1$ para el jugador $1$ es siempre el mínimo de la columna, para cualquier acción $j$ del jugador $2$ tendremos que $a_{i^\tau, j} > a_{n+1,j}$ y por tanto podemos afirmar que $i^\tau \in BR_1(y^{\tau - 1})$ en $A'$.

    Para $j^\tau$ podemos razonar de forma análoga, solo que considerando la pertenencia a $BR_2(x^{\tau - 1})$ o $BR_2(x^\tau)$ según si hablamos de juego ficticio secuencial o alternante.

    Claramente $(i^\tau, j^\tau)_{\tau \in \mathbb{N}}$ es también una secuencia de juego ficticio  válida en $A'$.
\end{proof}


\begin{lemma}
    Sea $A$ un juego intereses idénticos de tamaño $n \times m$, existe un juego de intereses idénticos $A'$ de tamaño $(n+1) \times m$ tal que toda secuencia $(i^\tau, j^\tau)_{\tau \in \mathbb{N}}$ de juego ficticio (secuencial o alternante) en $(A, B)$ lo es también en $(A', B')$.
\end{lemma}
\begin{proof}
    % Creo que vale la misma demostración de suma cero???
    % Por eso quería demostrar simétricos???
    Definimos el juego $A'$ como
    \begin{equation*}
        A' =
        \begin{pmatrix}
        a_{1,1} & a_{1,2} & \cdots & a_{1,m}     \\
        a_{2,1} & a_{2,2} & \cdots & a_{2,m}    \\
        \vdots  & \vdots  & \ddots & \vdots                  \\
        a_{n,1} & a_{n,2} & \cdots & a_{n,m}    \\
        \lambda & \lambda & \cdots &\lambda
        \end{pmatrix}
    \end{equation*} \falta{cambiar esto a las matrices nuevas}

    con $\lambda = min(A) - 1$.
    
    Sea $(i^\tau, j^\tau)_{\tau \in \mathbb{N}}$ una secuencia de juego ficticio secuencial sobre A, con secuencias de creencias $(x^\tau, y^\tau)_{\tau \in \mathbb{N}}$. En la iteración $\tau$, sabemos que $i^\tau \in BR_1(y^{\tau - 1})$ en $A$. Como el pago de $n+1$ para el jugador $1$ es siempre el mínimo de la columna, para cualquier acción $j$ del jugador $2$ tendremos que $a_{i^\tau, j} > a_{n+1,j}$ y por tanto podemos afirmar que $i^\tau \in BR_1(y^{\tau - 1})$ en $A'$.
\end{proof}


Observemos que no valen lemas similares para la reducción en tamaño de los juegos. Agregar estados con poco incentivo no modifica las secuencias de juego ficticio, pero para un juego puede haber secuencias que pasen por todas las acciones de cada un jugador y que por tanto este no pueda reducirse a uno más chico.

\section{Convergencia de Juego Ficticio} \label{sec:convergencia:fp}

% Mencionar que agregamos reglas de desempate y entonces el teorema es mas piola
En su publicación de 1997, Monderer y Sela \cite{no:cycling} enuncian un resultado que nos da una intuición interesante sobre el comportamiento del juego ficticio. Al ser el este un mecanismo utilizado para encontrar equilibrios de Nash, uno esperaría observar un comportamiento estable alrededor de los mismos. Monderer y Sela llaman a esto el Principio de Estabilidad. La demostración que presentan es mediante otros principios y conceptos que desarrollan en esa publicación, pero este resultado puede probarse de forma más directa. Presentaremos entonces a continuación una demostración alternativa.

\begin{lemma}
    Sea $(i^\tau, j^\tau)_{\tau \in \mathbb{N}}$ una secuencia de aprendizaje de juego ficticio simultáneo en el juego en forma bimatricial $(A, B)$ de tamaño $n \times m$, con secuencias de creencias $(x^\tau, y^\tau)_{\tau \in \mathbb{N}}$ y reglas de desempate $(d_1, d_2)$. Si en la iteración $k$ se jugó el equilibrio de Nash puro $(i^*, j^*)$, entonces $i^* \in BR_1(y^{k})$ y $j^* \in BR_2(x^{k})$.
\end{lemma}
\begin{proof}

    % sacar una desigualdad copada
    Comencemos con el caso del jugador fila. Queremos probar que $i^*$ es una mejor respuesta a las creencias del jugador fila sobre la estrategia del jugador columna. Veamos primero entonces que forma tienen estas creencias según como se actualizan.
    
    \[
        y^k = \frac{\sum_{s=1}^{k-1}{\pstrat{j^s}} + \pstrat{j^*}}{k}
            = \frac{\sum_{s=1}^{k-1}{\pstrat{j^s}}}{k} + \frac{\pstrat{j^*}}{k}
            = \frac{\sum_{s=1}^{k-1}{\pstrat{j^s}}(k-1)}{k(k-1)} + \frac{\pstrat{j^*}}{k}
            = \frac{k - 1}{k} y^{k-1} + \frac{\pstrat{j^*}}{k}
    \]
    Luego, tendremos que el conjunto de mejor respuesta será
    
    \begin{align*}
        BR_1(y^k) &= BR_1(\frac{k - 1}{k} y^{k-1} + \frac{j^*}{k}) \\
        &= \argmax_{i \in N} \{\pstrat{i} A(\frac{(k - 1)y^{k-1}}{k} + \frac{j^*}{k})\}\\
        &= \argmax_{i \in N} \{\frac{(k - 1)}{k}\pstrat{i} Ay^{k-1} + \frac{1}{k}\pstrat{i} Aj^*\}
    \end{align*}

    Como en la iteración $k$ se jugó el perfil $(i^*, j^*)$, por la definición \ref{def:fp:berger_desempate}, sabemos que $i^* \in BR_1(y^{k-1}) = \argmax_{i \in N}\{\pstrat{i}Ay^{k-1}\}$. Es decir que para cualquier $i \in N$, podemos afirmar que $\pstrat{i^*}Ay^{k-1} \geq \pstrat{i}Ay^{k-1}$ y también (multiplicando en ambos lados por una constante positiva) que $\frac{(k - 1)}{k}\pstrat{i^*}Ay^{k-1} > \frac{(k - 1)}{k}\pstrat{i}Ay^{k-1}$.

    Sabemos también, al ser un equilibrio de Nash, que $i^* \in BR_1(\pstrat{j^*}) = \argmax_{i \in N}\{\pstrat{i}Aj^*\}$. Es decir que para cualquier $i \in N$, $\pstrat{i^*}Aj^* \geq \pstrat{i}Aj^*$ y consecuentemente $\frac{1}{k}\pstrat{i^*}Aj^* \geq \frac{1}{k}\pstrat{i}Aj^*$.

    Podemos sumar estas dos desigualdades para afirmar que $\frac{(k - 1)}{k}\pstrat{i^*}Ay^{k-1} + \frac{1}{k}\pstrat{i^*}Aj^* \geq \frac{(k - 1)}{k}\pstrat{i}Ay^{k-1} + \frac{1}{k}\pstrat{i}Aj^*$ para cualquier $i \in N$ y por lo tanto $i^* \in BR_1(y^{k})$.

    Razonando análogamente para el jugador columna, podemos afirmar también que $j^* \in BR_2(x^{k})$.
    
\end{proof}

\begin{lemma}
    Sea $(i^\tau, j^\tau)_{\tau \in \mathbb{N}}$ una secuencia de aprendizaje de juego ficticio simultáneo en el juego en forma bimatricial $(A, B)$ de tamaño $n \times m$, con secuencias de creencias $(x^\tau, y^\tau)_{\tau \in \mathbb{N}}$ y reglas de desempate $(d_1, d_2)$. Si en la iteración $k$ se jugó el equilibrio de Nash puro $(i^*, j^*)$, entonces $BR_1(y^{k}) \subseteq BR_1(y^{k-1})$ y $BR_2(x^{k}) \subseteq BR_2(x^{k-1})$.
\end{lemma}

\begin{proof}
    Comencemos por el caso del jugador fila. Para probar que $BR_1(y^{k}) \subseteq BR_1(y^{k-1})$, debemos probar que para todo $i \in N$, se cumple que $i \in BR_1(y^{k}) \implies i \in BR_1(y^{k-1})$, o por contra-recíproco, que $i \notin BR_1(y^{k-1}) \implies i \notin BR_1(y^{k})$. Supongamos entonces un $i' \in N$ tal que $i' \notin BR_1(y^{k-1})$.

    Como $BR_1(y^{k-1}) = \argmax_{i \in N} \{\pstrat{i}Ay^{k-1}\}$, si $i^* \in BR_1(y^{k-1})$ pero $i' \notin BR_1(y^{k-1})$, entonces podemos afirmar que $\pstrat{i^*}Ay^{k-1} > \pstrat{i'}Ay^{k-1}$ y luego (multiplicando ambos lados por una constante positiva) que $\frac{(k - 1)}{k} \pstrat{i^*}Ay^{k-1} > \frac{(k - 1)}{k} \pstrat{i'}Ay^{k-1}$.
    
    Además, como $(i^*, j^*)$ es equilibrio de Nash puro, sabemos que $i^* \in BR_1(\pstrat{j*}) = \argmax_{i \in N} \{\pstrat{i}Aj^*\}$. Es decir que $\pstrat{i^*}Aj^* \ge \pstrat{i'}Aj^*$ y también $\frac{k - 1}{k} \pstrat{i^*}A\pstrat{j^*} \ge \frac{k - 1}{k} \pstrat{i'}A\pstrat{j^*}$.

    Entonces, podemos razonar de la siguiente manera:
    \begin{align*}
        \pstrat{i^*}Ay^k &= \pstrat{i^*}A(\frac{k - 1}{k} y^{k-1} + \frac{\pstrat{j^*}}{k}) = \frac{(k - 1)}{k} \pstrat{i^*}Ay^{k-1} + \frac{1}{k}\pstrat{i^*}A\pstrat{j^*}\\
        &> \frac{(k - 1)}{k} \pstrat{i'}Ay^{k-1} + \frac{1}{k}\pstrat{i'}A\pstrat{j^*} = \pstrat{i'}A(\frac{k - 1}{k} y^{k-1} + \frac{\pstrat{j^*}}{k}) = \pstrat{i'}Ay^k
    \end{align*}

    Puesto que sabemos por el lema anterior, que $i^* \in BR_1(y^{k}) = \argmax_{i \in N} \{\pstrat{i}Ay^{k}\}$, podemos afirmar que $i' \notin BR_1(y^{k})$.
    Podemos razonar análogamente para el jugador columna para demostrar también que $BR_2(x^{k}) \subseteq BR_2(x^{k-1})$.
\end{proof}

\begin{theorem}
    Sea $(i^\tau, j^\tau)_{\tau \in \mathbb{N}}$ una secuencia de aprendizaje de juego ficticio simultáneo en el juego en forma bimatricial $(A, B)$ de tamaño $n \times m$, con secuencias de creencias $(x^\tau, y^\tau)_{\tau \in \mathbb{N}}$ y reglas de desempate $(d_1, d_2)$. Si en la iteración $k$ se jugó el equilibrio de Nash puro $(i*, j*)$, entonces este perfil se repetirá en la iteración siguiente.
\end{theorem}
\begin{proof}
    Nuevamente, comencemos por el jugador fila. Sabemos que en la iteración $k$ se jugó $(i*, j*)$, lo cual nos asegura que $d_1(BR_1(y^{k-1})) = i^*$. Pero sabemos también por los lemas anteriores que $i^* \in BR_1(y^{k})$ y que $BR_1(y^{k}) \subseteq BR_1(y^{k-1})$. Luego, por la definición de reglas de desempate, debe ser que $d_1(BR_1(y^k)) = i^*$. Es decir, volverá a jugar $i^*$.

    Análogamente, podemos afirmar también que el jugador columna jugará $j^*$.
\end{proof}


\section{Velocidad de Convergencia de AFP} \label{sec:aportes:velocidad}

En esta sección nos enfocaremos en estudiar más detalladamente los resultados de Brandt, Fischer y Harrenstein \cite{brandt:rate:convergence}. En su paper, los autores presentan cotas superiores para la velocidad de convergencia del juego ficticio simultáneo en clases de juegos que se sabe que siempre convergen de forma pura \falta{checkear esto}. Como dijimos en el capítulo \ref{cap:relwork}, los autores mencionan la posibilidad de expandir sus resultados a la variante alternante de juego ficticio. Presentamos a continuación nuestro análisis para los casos de los juegos de suma constante simétricos y los no degenerados de intereses idénticos.

% Pero antes, nos parece oportuno hablar mas detalladamente sobre algunas cuestiones respecto a las métricas contra las que se estudia la velocidad de convergencia de un proceso de aprendizaje en juegos. Puede llamar la atención que si bien en toda la bibliografía previa al trabajo de Brandt, Fischer y Harrenstein se habla sobre la velocidad de convergencia en función del tamaño del juego, los autores prefieren estudiarla en función del tamaño de representación en bits de los valores de las utilidades del juego. Este tipo de resultados es en realidad más general. Observemos que un juego que tenga $k$ acciones para un jugador necesitará, en cualquier codificación capaz de representar juegos bimatriciales finitos arbitrarios, $O(k)$ bits, ya que deberá codificar cada una de estas $k$ acciones con al menos un bit. Por otro lado, a efectos del estudio de la convergencia del juego ficticio, un juego con conjuntos de acciones de tamaños menores a $k$ pero que requiere $O(k)$ bits para ser representado, puede expandirse a uno con comportamiento equivalente de tamaño $k$. Esto lo mostraremos en los siguientes lemas: \falta{hay que retocar este parrafo para referenciar la discusion previa sobre tamaños, bits, etc}

% \falta{comentar la cuestion de que se buscan caso especificos porque se generaliza a clases mas grandes}

Veamos primero que para el caso de los juegos de suma constante simétricos, el teorema de estos autores efectivamente es expandible de forma bastante directa a la variante alternante. Vale la pena notar que formulamos el teorema en términos de que el último perfil no sea un equilibrio de Nash puro, en vez de pedir que ninguno en la secuencia lo sea como hacen Brandt, Fischer y Harrenstein ya que por el principio de estabilidad, si alguno de los perfiles jugados en la secuencia fuera un equilibrio de Nash puro, todos los siguientes lo serían.

\begin{theorem} \label{teorema:afp:velocidad:simetricos}
    Existe un juego simétrico $A$ representable en $O(k)$ bits, con al menos un equilibrio de Nash puro y una secuencia de juego ficticio alternante $(i^\tau, j^\tau)_{\tau \in \mathbb{N}}$ sobre $A$ tal que $(i^{2^k}, j^{2^k})$ no es un equilibrio de Nash puro.
\end{theorem}
\begin{proof}
    Consideremos un juego en forma bimatricial con la siguiente matriz de pagos:

    \begin{center}
    \begin{tabular}{llll}
    \                           & $j_1$      & $j_2$      & $j_3$                            \\ \cline{2-4}
    \multicolumn{1}{l|}{$i_1$} & 0          & -1         & \multicolumn{1}{l|}{$-\epsilon$} \\
    \multicolumn{1}{l|}{$i_2$} & 1          & 0          & \multicolumn{1}{l|}{$-\epsilon$} \\
    \multicolumn{1}{l|}{$i_3$} & $\epsilon$ & $\epsilon$ & \multicolumn{1}{l|}{0}           \\ \cline{2-4}
    \end{tabular}
\end{center}

    Si $\epsilon < 1$, vemos que $(i^3, j^3)$ es el único equilibrio de Nash puro por ser el único perfil restante luego de realizar eliminación iterada de estrategias estrictamente dominadas. \falta{Esto no explicamos qué es. Supongo que a los conceptos previos?}

    Consideremos un número $k > 1$ arbitrario y sea $\epsilon = 2^{-k}$. Para estos valores, $\epsilon$ puede codificarse en $O(k)$ bits, mientras que las otras utilidades del juego son constantes, por lo que podemos afirmar que la representación del juego será tambien del orden de  $O(k)$ bits. Por lo tanto, si probamos que un proceso de juego ficticio alternante puede requerir $2^k$ rondas antes de que se juegue $(i^3, j^3)$, el teorema estará demostrado.

    Si el proceso comienza con el jugador fila jugando $i^1$, entonces las utilidades esperadas del jugador columna serán $-x^1A = (0, 1, \epsilon)$ y elegirá $j_2$.

    En la siguiente ronda, el jugador $1$ reaccionará con $i_3$, dando que el jugador $2$ tendrá una creencia sobre su estrategia de $x^2 = (\frac{1}{2},0,\frac{1}{2})$, por lo que las utilidades esperadas del jugador $2$ serán $-x^2A = (\frac{-\epsilon}{2}, \frac{1-\epsilon}{2}, \frac{\epsilon}{2})$y volverá a elegir $j_2$.

    Este perfil $(i_3, j_2)$ se repetirá $2^k - 1$ rondas, ya que tendremos que mientras $2 \le \tau \le 2^k$, se cumplirán:

    \begin{align*}
        x^\tau     &= \frac{\pstrat{i_1} + (\tau - 1) \pstrat{i_3}}{\tau} = (\frac{1}{\tau}, 0,\frac{\tau - 1}{\tau}) \\
        -x^{\tau}A &= (-\frac{(\tau-1) \epsilon}{\tau}, \frac{1-(\tau-1)\epsilon}{\tau}, \frac{\epsilon}{\tau}) \\
        y^\tau     &= \pstrat{j_2} = (0, 1, 0) \\
        Ay^\tau    &= (- 1, 0, \epsilon) \\
    \end{align*}

    \begin{table}
        \begin{tabular}{llllll}
    \toprule
    {} &       $(i^\tau, j^\tau)$ &             $x^\tau$ &               $x^{\tau}B$ &                $y^\tau$ &                 $Ay^\tau$ \\
    $\tau$ &                &                         &                           &                         &                           \\
    \midrule
    $1$         &  $(i_1,\ j_2)$ &  $(1,\ 0,\ 0)$                                   & $(0, 1, 2^{-k})$                                                                          &  $(0,\ 1,\ 0)$    &   $(-1,\ 0,\ 2^{-k})$ \\
    $2$         &  $(i_3,\ j_2)$ &  $(\frac{1}{2},\ 0,\ \frac{1}{2})$               & $(-2^{-(k+1)},\ \frac{1 - 2^{-k}}{2},\ 2^{-(k+1)})$                                        &  $(0,\ 1,\ 0)$    &   $(-1,\ 0,\ 2^{-k})$ \\
    $3$         &  $(i_3,\ j_2)$ &  $(\frac{1}{3},\ 0,\ \frac{2}{3})$               & $(-\frac{2^{-k+1}}{3},\ \frac{1 - 2^{-k+1}}{3},\ \frac{2^{-k}}{3})$                       &  $(0,\ 1,\ 0)$    &   $(-1,\ 0,\ 2^{-k})$ \\
                &  $\vdots$      &   &   &   &   \\
    $\tau$      &  $(i_3,\ j_2)$ &  $(\frac{1}{\tau},\ 0,\ \frac{\tau - 1}{\tau})$  & $(-\frac{(\tau-1)2^{-k}}{\tau},\ \frac{1 - (\tau-1)2^{-k}}{\tau},\ \frac{2^{-k}}{\tau})$  &  $(0,\ 1,\ 0)$    &   $(-1,\ 0,\ 2^{-k})$ \\
                &  $\vdots$      &   &   &   &   \\
    $2^k$       &  $(i_3,\ j_2)$ &  $(\epsilon,\ 0,\ \frac{\epsilon-1}{\epsilon})$           & $(\epsilon (\epsilon-1),\ \epsilon^2,\ \epsilon^2)$                                                     &  $(0,\ 1,\ 0)$    &   $(-1,\ 0,\ 2^{-k})$\\
    \bottomrule
    \\
    \end{tabular}
        \caption{Proceso de juego ficticio alternante en el juego del teorema \ref{teorema:afp:velocidad:simetricos}}
        \label{tabla:afp:velocidad:simetricos}
    \end{table}

    La tabla \ref{tabla:afp:velocidad:simetricos}, muestra como se desarrolla este proceso. Para el jugador fila, justificar su decisión es trivial ya que la estrategia percibida de su oponente es pura e $i_3$ es la única acción con utilidad esperada positiva. Para entender por qué el jugador columna no cambia su estrategia, debemos notar que:

    \begin{align*}
        \tau &\le 2^k = \frac{1}{\epsilon} \\
        \epsilon \tau&\le 1 \\
        \epsilon (\tau - 1 + 1) &\le 1\\
        (\tau - 1) \epsilon + \epsilon &\le 1\\
        1-(\tau-1)\epsilon &\ge \epsilon \\
        \frac{1-(\tau-1)\epsilon}{\tau} &\ge \frac{\epsilon}{\tau} \\
    \end{align*}

    Esto podemos interpretarlo como que si bien en las iteraciones recientes el jugador $1$ jugó $i_3$, el incentivo resultante de la única vez que jugó $i_1$ es muy fuerte por la gran diferencia de utilidades para el jugador $2$ entre $(i_1, j_2)$ e $(i_2, j_3)$, por lo que deberán pasar $2^{k-1}$ iteraciones de $i_3$ luego de ese único $i_1$ para que las utilidades esperadas se compensen.

    Concluimos entonces que la secuencia

    \begin{center}
    \begin{math}
        (i_1, j_2), \underbrace{(i_3, j_2), ... (i_3, j_2)}_{\text{$2^k - 1$ veces}}
    \end{math}
    \end{center}

    es una secuencia de aprendizaje de juego ficticio alternante válida de este juego que es exponencialmente larga en $k$ y en la cual no se juega ningún equilibrio de Nash puro.

\end{proof}

Por su parte, la demostración para el caso de los juegos no degenerados de $2 \times 3$ es un poco menos directa y requiere plantear una ligera variante del juego originalmente propuesto por Brandt, Fischer y Harrenstein. \falta{discutir un poco como afecta el cambio de la matriz}

\begin{theorem} \label{teorema:afp:velocidad:nondegen}
    Existe un juego no degenerado de intereses idénticos $A$ representable en $O(k)$ bits, con al menos un equilibrio de Nash puro y una secuencia de juego ficticio alternante $(i^\tau, j^\tau)_{\tau \in \mathbb{N}}$ sobre $A$ tal que para todo $\tau < 2^k$, $(i^\tau, j^\tau)$ no es un equilibrio de Nash puro.
\end{theorem}

\begin{proof}
    Consideremos un juego en forma bimatricial con la siguiente matriz de pagos:

    \begin{center}
    \begin{tabular}{llll}
    \                           & $j_1$    & $j_2$                      & $j_3$                           \\ \cline{2-4}
    \multicolumn{1}{l|}{$i_1$} & $1$ & $2$          & \multicolumn{1}{l|}{$0$} \\
    \multicolumn{1}{l|}{$i_2$} & $0$ & $2+\epsilon$ & \multicolumn{1}{l|}{$2+2\epsilon$} \\ \cline{2-4}
    \end{tabular}
\end{center}

    Si $\epsilon < 1$, vemos que $(i^2, j^3)$ es el único equilibrio de Nash puro por ser el único perfil restante luego de realizar eliminación iterada de estrategias estrictamente dominadas.

    Consideremos un número $k > 1$ arbitrario. Mostraremos que para $\epsilon = 2^{-k}$, un proceso de juego ficticio alternante puede tomar $2^k$ rondas antes de que se juegue $(i^2, j^3)$. Al igual que en la demostración anterior, el juego puede codificarse en $O(k)$ bits, por lo que esto demuestra el teorema.

    Si el proceso comienza con el jugador fila jugando $i^1$, entonces $x^1B = (1, 2, 0)$ y por lo tanto el jugador columna elegirá $j_2$.

    En la siguiente ronda, el jugador $1$ reaccionará con $i_2$, dando que el jugador $2$ tendrá una creencia sobre su estrategia de $x^2 = (\frac{1}{2},\frac{1}{2})$, por lo que las utilidades esperadas del jugador $2$ serán $-x^2A = (\frac{1}{2}, 2 + \frac{\epsilon}{2}, 1 + \epsilon)$ y volverá a elegir $j_2$.

    Este perfil $(i_2, j_2)$ se repetirá $2^k - 1$ rondas, ya que tendremos que mientras $2 \le \tau \le 2^k$, se cumplirán:

    \begin{align*}
        x^\tau     &= \frac{\pstrat{i_1} + (\tau - 1) \pstrat{i_2}}{\tau} = (\frac{1}{\tau}, \frac{\tau - 1}{\tau}) \\
        x^{\tau}A &= (\frac{1}{\tau}, \frac{2 + (\tau - 1)(2 + \epsilon)}{\tau}, \frac{(\tau - 1)(2 + 2 \epsilon)}{\tau}) \\
        y^\tau     &= \pstrat{j_2} = (0, 1, 0) \\
        Ay^\tau    &= (2, 2 + \epsilon) \\
    \end{align*}

    \begin{table}
        \begin{tabular}{llllll}
\toprule
{} &       $(i, j)$ &              $x$ &                    $x{^\tau}B$ &                     $y$ &             $Ay$ \\
Ronda   &                &                  &                         &                         &                  \\
\midrule
1       &  $(i_1,\ j_2)$ &  $(1, 0)$                                    &  $(1, 2, 0)$                                                                                              &  $(0, 1, 0)$ &  $(2, 2 + \epsilon)$ \\
2       &  $(i_2,\ j_2)$ &  $(\frac{1}{2}, \frac{1}{2})$                &  $(\frac{1}{2}, 2 + \frac{\epsilon}{2}, 1 + \epsilon)$                                                    & $(0, 1, 0)$   & $(2, 2 + \epsilon)$ \\
3       &  $(i_2,\ j_2)$ &  $(\frac{1}{3}, \frac{2}{3})$                &  $(\frac{1}{3}, 2 + \frac{2}{3} \epsilon, \frac{4 + 4 \epsilon)}{3})$   & $(0, 1, 0)$   & $(2, 2 + \epsilon)$ \\
        &  \vdots       \\
$\tau$  &  $(i_2,\ j_2)$ &  $(\frac{1}{\tau}, \frac{\tau - 1}{\tau})$   &  $(\frac{1}{\tau}, \frac{2 + (\tau - 1)(2 + \epsilon)}{\tau}, \frac{(\tau - 1)(2 + 2 \epsilon)}{\tau})$   & $(0, 1, 0)$   & $(2, 2 + \epsilon)$ \\
        &  \vdots       \\
$2^k$   &  $(i_2,\ j_2)$ &  $(\epsilon, 1 - \epsilon)$   &  $(\epsilon, 1 - \epsilon - \epsilon^2, 2 - \epsilon^2)$   & $(0, 1, 0)$   & $(2, 2 + \epsilon)$ \\
\bottomrule
\end{tabular}

        \caption{Proceso de juego ficticio alternante en el juego del teorema \ref{teorema:afp:velocidad:nondegen}}
        \label{tabla:afp:velocidad:nondegen}
    \end{table}

    La tabla \ref{tabla:afp:velocidad:nondegen}, muestra como se desarrolla este proceso. Para el jugador fila, justificar su decisión es trivial ya que la estrategia percibida de su oponente es pura y la utilidad esperada de  $i_2$ es siempre marginalmente mayor que la de $i_1$. Para entender por qué el jugador columna no cambia su estrategia, debemos notar que:

    \begin{align*}
        \tau &\le 2^k = \frac{1}{\epsilon} \\
        \tau &\le \frac{2}{\epsilon} + 1 \\
        \frac{2}{\epsilon} &\ge \tau - 1 \\
        \frac{2}{\tau - 1} &\ge \epsilon \\
        \frac{2}{(\tau - 1)} + 2 + \epsilon &\ge 2 + 2 \epsilon \\
        2 + (\tau - 1) (2 + \epsilon) &\ge (\tau - 1) (2 + 2 \epsilon) \\
    \end{align*}

    Similarmente al teorema anterior, esto podemos interpretarlo como que si bien en las iteraciones recientes el jugador $1$ jugó $i_2$, el incentivo resultante de la única vez que jugó $i_1$ es muy fuerte por la gran diferencia de utilidades para el jugador $2$ entre $(i_1, j_2)$ e $(i_2, j_3)$ \falta{checkear esto}, por lo que deberán pasar $2^{k-1}$ iteraciones de $i_3$ luego de ese único $i_1$ para que las utilidades esperadas se compensen.

    Concluimos entonces que la secuencia

    \begin{center}
    \begin{math}
        (i_1, j_2), \underbrace{(i_2, j_2), ... (i_2, j_2)}_{\text{$2^k - 1$ veces}}
    \end{math}
    \end{center}

    es una secuencia de aprendizaje de juego ficticio alternante válida de este juego que es exponencialmente larga en $k$ y en la cual no se juega ningún equilibrio de Nash puro.
\end{proof}

El detalle de que el juego originalmente propuesto por Brandt, Fischer y Harrenstein no nos sirva para demostrar el teorema anterior no es para nada menor. En efecto, como veremos en el siguiente teorema, es un ejemplo de un juego en el un proceso de juego ficticio simultáneo puede requerir una cantidad de rondas exponenciales mientras que, toda secuencia de juego ficticio alternante convergerá rápidamente.

\begin{theorem} \label{teorema:afp:mejor}
    Existe un juego no degenerado de intereses idénticos $A$ representable en $O(k)$ bits y una secuencia de juego ficticio simultáneo $(i^\tau, j^\tau)_{\tau \in \mathbb{N}}$ sobre $A$ y una constante $C$ tal que
    \begin{itemize}
        \item Para todo $\tau < 2^k$, $(i^\tau, j^\tau)$ no es un equilibrio de Nash puro.
        \item Para todo proceso de juego ficticio alternante $(\widehat{i}^\tau, \widehat{j}^\tau)_{\tau \in \mathbb{N}}$ en $A$, $(\widehat{i}^C, \widehat{j}^C)$ es un equilibrio de Nash puro.
    \end{itemize}
\end{theorem}

% \begin{theorem}
%     Existe un juego para el cuál un proceso de juego ficticio simultáneo puede requerir una cantidad exponencial (en el tamaño de representación en bits de los valores de las utilidades del juego) de rondas antes de que un equilibrio sea jugado, mientras que en todo proceso de juego ficticio alternado se jugará un equilibrio en una cantidad de rondas acotada por una constante.
% \end{theorem}

\begin{proof}
    Consideremos el juego en forma bimatricial con la siguiente matriz de pagos:

    \begin{center}
    \begin{tabular}{llll}
    .                          & $j^1$    & $j^2$                      & $j^3$                           \\ \cline{2-4}
    \multicolumn{1}{l|}{$i^1$} & $1$ & $2$                   & \multicolumn{1}{l|}{$0$} \\
    \multicolumn{1}{l|}{$i^2$} & $0$ & $2+\epsilon$ & \multicolumn{1}{l|}{$3$} \\ \cline{2-4}
    \end{tabular}
\end{center}

    Si $\epsilon < 1$, $(i^2, j^3)$ es el único equilibrio de Nash puro por ser el único perfil restante luego de realizar eliminación iterada de estrategias estrictamente dominadas.

    Consideremos un número $k > 1$ arbitrario. Mostraremos que para $\epsilon = 2^{-k}$, un proceso de juego ficticio simultáneo puede tomar $2^k$ rondas antes de que se juegue $(i^2, j^3)$, mientras que todo proceso de juego ficticio alternante converge de forma pura en un número constante de rondas. Al igual que los teoremas anteriores, el juego puede codificarse en $O(k)$ bits, por lo que esto demuestra el teorema.

    Veamos primero el caso alternante. Existen dos posibles secuencias de juego ficticio alternante para este juego, dado que el jugador $1$ elegirá primero y el jugador $2$ reaccionara según esta decisión.

    \begin{table}
        \centering
        \begin{tabular}{llllll}
    \toprule
    {} &       $(i, j)$ &              $x$ &                    $xB$ &                     $y$ &             $Ay$ \\
    Iteración &                &                  &                         &                         &                  \\
    \midrule
    $1$         &  $(i_2, j_3)$ &  $(0, 1)$ &  $(0, 2 + \epsilon, 3)$ &  $(0, 0, 1)$ &  $(0 , 3)$\\
    \bottomrule
    \\
    \end{tabular}
        \caption{Proceso de juego ficticio alternante sobre el juego del teorema \ref{teorema:afp:mejor} comenzando por $i_2$}
        \label{tabla:afp:mejor:b}
        \centering
        \begin{tabular}{llllll}
    \toprule
    {} &       $(i^\tau, j^\tau)$ &             $x^\tau$ &               $x^{\tau}B$ &                $y^\tau$ &                 $Ay^\tau$ \\
    $\tau$ &                &                         &                           &                         &                           \\
    \midrule
    $1$         &  $(i_1,\ j_2)$ &  $(1, 0)$ &  $(1, 2, 0 )$ &  $(0, 1, 0)$ &  $(2, 2 + \epsilon)$ \\
    $2$         &  $(i_2,\ j_2)$ &  $(\frac{1}{2}, \frac{1}{2})$ &  $(\frac{1}{2}, 2 + \frac{\epsilon}{2}, \frac{3}{2})$ &  $(0, 1, 0)$ &  $(2, 2 + \epsilon)$ \\
    $3$         &  $(i_2,\ j_2)$ &  $(\frac{1}{3}, \frac{2}{3})$ &  $(\frac{1}{3}, 2 + \frac{2}{3}\epsilon, 2)$ &  $(0, 1, 0)$ &  $(2, 2 + \epsilon)$ \\
    $4$         &  $(i_2,\ j_3)$ &  $(\frac{1}{4}, \frac{3}{4})$ &  $(\frac{1}{4}, 2 + \frac{3}{4}\epsilon, \frac{9}{4})$ &  $(0, \frac{3}{4}, \frac{1}{4})$ &  $(\frac{3}{2}, \frac{3(3+\epsilon)}{4})$\\
    \bottomrule
    \\
    \end{tabular}
        \caption{Proceso de juego ficticio alternante sobre el juego del teorema \ref{teorema:afp:mejor} comenzando por $i_1$}
        \label{tabla:afp:mejor:a}
        \centering
        \begin{tabular}{llllll}
\toprule
{} &       $(i, j)$ &              $x$     $x{^\tau}B$ &    $y$ &             $Ay$ \\
Iteración &                &                  &                         &                         &                  \\
\midrule
$1$         &  $(i_1, j_1)$ &  $(1, 0)$ &  $(1, 2, 0)$ &  $(1, 0, 0)$ &                                $(1 , 0)$ \\
$2$         &  $(i_1, j_2)$ &  $(1, 0)$ &  $(1, 2, 0)$ &  $(\frac{1}{2}, \frac{1}{2}, 0)$ &              $(\frac{3}{2}, \frac{2 + \epsilon}{2})$ \\
$3$         &  $(i_1, j_2)$ &  $(1, 0)$ &  $(1, 2, 0)$ &  $(\frac{1}{3}, \frac{2}{3}, 0)$ &              $(0 , 3)$ \\
        &  \vdots       \\
$\tau$  &  $(i_1,\ j_2)$    &  $(1, 0)$ &  $(1, 2, 0)$  & $(\frac{1}{\tau}, \frac{\tau - 1}{\tau}, 0)$ & $(2 - \frac{1}{\tau}, 2 - \frac{2 + \epsilon}{\tau})$ \\
        &  \vdots       \\
$2^k$   &  $(i_1,\ j_2)$ &     $(1, 0)$ &  $(1, 2, 0)$  & $(\frac{1}{2^k}, \frac{2^k-1}{2^k}, 0)$      & $(2 - \frac{1}{2^k}, 2 - \frac{2 + \epsilon}{2^k})$ \\
            \bottomrule
\end{tabular}

        \caption{Proceso de juego ficticio simultáneo sobre el juego del teorema \ref{teorema:afp:mejor} comenzando por $(i_1, j_1)$}
        \label{tabla:afp:mejor:c}
    \end{table}

    Si el jugador fila juega $i^2$, el jugador columna responderá con $j^3$, siendo este el equilibrio puro. Si el jugador fila comienza con $i^1$, el jugador columna responderá con $j^2$. Esto hará que el jugador fila juegue $i^2$ en la segunda ronda por ser $A\pstrat{j^2} = (2, 2 + \epsilon)$, mientras que el jugador continuará fila continuara jugando $j^2$. Esta situación se repetirá una ronda más, tras la cuál, la jugador columna se verá incentivado a jugar $j^3$. Los desarrollos de estos dos procesos pueden verse en las tablas \ref{tabla:afp:mejor:a} y \ref{tabla:afp:mejor:b}. Vemos entonces que, independientemente del valor de $k$, ambos procesos convergen de forma pura en 4 rondas o menos.

    Pasemos ahora al caso simultáneo, para el cual nos basaremos en la prueba de Brandt, Fischer Y Harrenstein. Si el proceso comienza con el perfil $(i^1, j^1)$, entonces las utilidades esperadas serán $Ay^1 = (1, 0)$ e $x^1 B = (1, 2, 0)$ respectivamente. Luego, en la segunda iteración el jugador fila elegirá $i^1$ y el jugador columna $j^2$. Las utilidades esperadas se actualizarán entonces como $Ay^2 = (\frac{3}{2}, \frac{2 + \epsilon}{2})$ y $x^2 B = (1, 2, 0)$.

    A continuación, por al menos $2^k$ rondas, los jugadores elegirán las mismas jugadas que en la iteración 2, dado que para todo $\tau$ tal que $2 \le i \le 2^k$, tendremos $Ay^\tau = (2 - \frac{1}{\tau}, 2 - \frac{2 + \epsilon}{\tau})$ e $x^\tau B = (1, 2, 0)$. La tabla \ref{tabla:afp:mejor:c} muestra como se desarrolla este proceso.

    Concluimos entonces que la secuencia de perfiles

    \begin{center}
    \begin{math}
        (i_1, j_1), \underbrace{(i_1, j_2), ... (i_1, j_2)}_{\text{$2^k$ veces}}
    \end{math}
    \end{center}


    es una secuencia de aprendizaje de juego ficticio simultáneo válida de este juego que es exponencialmente larga en $k$ y en la cual no se juega ningún equilibrio de Nash puro.

\end{proof}