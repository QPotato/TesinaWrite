\chapter{Resultados}  \label{cap:aportes}

\section{Equivalencia de las distintas definiciones}

Como mencionamos en la sección \ref{sec:def:fp}, existen dos formas de definir el juego ficticio entre los distintos autores de la literatura. Ambas simplifican de distintas formas la definición original de Brown y si bien son similares, su equivalencia no es inmediatamente evidente, por lo que uno podría dudar de si un teorema expresado para una de las definiciones es válido con la otra. Por lo tanto, presentamos a continuación dos lemas sobre esta equivalencia, para el caso simultáneo y alternante respectivamente. La idea será probar que los historiales de la definición \ref{def:fp:brandt} suman en cada iteración la estrategia pura correspondiente a la acción elegida por la definición \ref{def:fp:berger}. Comenzamos con el caso simultáneo.

\begin{lemma} \label{lema:equiv:sim}
    Sea $(A, B)$ un juego en forma bimatricial de $n \times m$, $(\pstrat{i^\tau}, \pstrat{j^\tau})_{\tau \in \mathbb{N}}$ una secuencia de juego ficticio simultáneo (según la definición \ref{def:fp:berger}) con secuencias de creencias $y^\tau$, $x^\tau$ y sea $(p^\tau, q^\tau)_{\tau \in \mathbb{N}}$ una secuencia de juego ficticio simultáneo (según la definición \ref{def:fp:brandt}), tales que $p^1 = \pstrat{i^1}$, $q^1 = \pstrat{j^1}$ y ambas usan las mismas reglas de desempate $(d_1, d_2)$. Entonces, $(i^\tau, j^\tau)_{\tau \in \mathbb{N}}$ y $(p^\tau, q^\tau)_{\tau \in \mathbb{N}}$ representan el mismo proceso de aprendizaje. Es decir, $\forall \tau \in \mathbb{N}$, se cumplen:

    \[ p^{\tau} = \sum_{s=1}^\tau{\pstrat{i^s}} \]
    \[ q^{\tau} = \sum_{s=1}^\tau{\pstrat{j^s}} \]

\end{lemma}
\begin{proof}
    Procederemos por inducción sobre $\tau$.

    Para $\tau = 1$, tenemos $p^1 = \pstrat{i^1} = \sum_{s=1}^1{\pstrat{i^s}}$ y $q^1 = \pstrat{j^1} = \sum_{s=1}^1{\pstrat{j^s}}$.

    Veamos ahora el caso de un $\tau > 1$, suponiendo que $p^{\tau-1} = \sum_{s=1}^{\tau-1}{\pstrat{i^s}}$ y $q^{\tau-1} = \sum_{s=1}^{\tau-1}{\pstrat{j^s}}$.
    Por la definición \ref{def:fp:brandt}, $p^\tau = p^{\tau-1} + \pstrat{i}$
    donde $i = d_1(argmax_{i \in N}\{\pstrat{i}Ap^{\tau-1}\})$.
    Pero también sabemos, por la definición \ref{def:fp:berger}
    que $i^\tau = d_1(BR_1(y^{\tau-1})) = d_1(BR_1(\frac{\sum_{s=1}^{\tau-1} \pstrat{j^s}}{\tau-1})) = d_1(argmax_{i \in N}\{A\frac{\sum_{s=1}^{\tau-1} \pstrat{j^s}}{\tau-1})\}) = d_1(argmax_{i \in N}\{A\frac{q^{\tau-1}}{\tau-1})\})$.
    Como escalar un vector no afecta la relación de orden entre sus componentes,
    podemos afirmar también que $i^\tau = d_1(argmax_{i \in N}\{Aq^{\tau-1})\})$.
    Luego, aplicando esto a la definición \ref{def:fp:brandt}, $p^\tau = p^{\tau-1} + \pstrat{i^\tau} = \sum_{s=1}^{\tau-1}{\pstrat{i^s}} + \pstrat{i^\tau} = \sum_{s=1}^{\tau}{\pstrat{i^s}}$.
    
    Análogamente, $q^{\tau} = \sum_{s=1}^\tau{\pstrat{j^s}}$.
\end{proof}

En el caso alternante, la jugada del jugador columna ya no es análoga a la del jugador fila, y el análisis es un poco mas complejo.

\begin{lemma}
    Sea $(A, B)$ un juego en forma bimatricial de $n \times m$, $(i^\tau, j^\tau)_{\tau \in \mathbb{N}}$ una secuencia de juego ficticio alternante (según la definición \ref{def:fp:berger}) con secuencias de creencias $y^\tau$, $x^\tau$ y sea $(p^\tau, q^\tau)_{\tau \in \mathbb{N}}$ una secuencia de juego ficticio alternante (según la definición \ref{def:fp:brandt}), tales que $p^1 = \pstrat{i^1}$ y ambas usan las mismas reglas de desempate $(d_1, d_2)$. Entonces, $(i^\tau, j^\tau)_{\tau \in \mathbb{N}}$ y $(p^\tau, q^\tau)_{\tau \in \mathbb{N}}$ representan el mismo proceso de aprendizaje. Es decir, para todo $\tau \in \mathbb{N}$, se cumplen:

    \[ p^{\tau} = \sum_{s=1}^\tau{\pstrat{i^s}} \]
    \[ q^{\tau} = \sum_{s=1}^\tau{\pstrat{j^s}} \]

\end{lemma}
\begin{proof}
    Nuevamente, procederemos por inducción sobre $\tau$.

    Para $\tau = 1$, sabemos que $p^1 = \pstrat{i^1} = \sum_{s=1}^1{\pstrat{i^s}}$.
    Por la definición \ref{def:fp:brandt}, $q^1 = q^0 + \pstrat{j}$ donde $q^0$ es el vector nulo y
    $j = d_2(argmax_{j \in M}\{p^{1}B\pstrat{j}\}) = d_2(argmax_{j \in M}\{\pstrat{i^{1}}B\pstrat{j}\})$.
    Además, por la definición \ref{def:fp:berger}, sabemos que
    $j^1 = d_2(BR_2(x^1)) = d_2(BR_2(\frac{\sum^1_{s=1} \pstrat{i^s}}{\tau})) = d_2(BR_2(i^1)) = d_2(argmax_{j \in M}\{\pstrat{i^{1}}B\pstrat{j}\})$.
    Luego, $q^1 = \pstrat{j^1} = \sum_{s=1}^1{\pstrat{j^s}}$.

    Veamos ahora el caso de $\tau > 1$, suponiendo que
    $p^{\tau-1} = \sum_{s=1}^{\tau-1}{\pstrat{i^s}}$ y $q^{\tau-1} = \sum_{s=1}^{\tau-1}{\pstrat{j^s}}$.
    Podemos afirmar, con el mismo argumento que en el caso inductivo del lema \ref{lema:equiv:sim}, que
    $p^{\tau} = \sum_{s=1}^\tau{\pstrat{i^s}}$.
    Por otro lado, $q^{\tau} = y^{\tau} + \pstrat{j}$ donde
    $j = d_2(argmax_{j' \in M}\{p^{\tau+1}B\pstrat{j'}\}) = d_2(argmax_{j' \in M}\{\sum_{s=1}^\tau{\pstrat{i^s}}B\pstrat{j'}\})$.
    Sabemos además, por la definición \ref{def:fp:berger}, que
    $j^\tau = d_2(BR_2(x^{\tau})) = d_2(BR_2(\frac{\sum_{s=1}^{\tau}{\pstrat{i^s}}}{\tau})) = d_2(argmax_{j' \in N}\{\frac{\sum_{s=1}^{\tau}{\pstrat{i^s}}}{\tau}B\pstrat{j}\})  = d_2(argmax_{j' \in N}\{\frac{p^{\tau}}{\tau}B\pstrat{j}\})$.
    Como escalar un vector no afecta la relación de orden entre sus componentes, podemos afirmar también que
    $j^\tau = d_2(argmax_{j' \in N}\{p^{\tau}B\pstrat{j}\})$.
    Luego, aplicando esto a la definición \ref{def:fp:brandt},
    $q^\tau = q^{\tau-1} + j^\tau = \sum_{s=1}^{\tau-1}{\pstrat{j^s}} + j^\tau = \sum_{s=1}^{\tau}{\pstrat{j^s}}$.

\end{proof}


\section{Preservación del Juego Ficticio}

Tratamos a continuación el analizar operaciones matriciales que preserven secuencias de juego ficticio, esto es, funciones $f$ sobre matrices que garanticen que, dada una matriz $A$ y una secuencia $s$ de juego ficticio cualquiera sobre $A$, la misma $s$ sea también una secuencia de juego ficticio sobre $f(A)$. En primer lugar, cualquier transformación que preserve las relaciones de orden de los pagos manteniendo el tamaño del juego (tales como escalar todos los pagos por un mismo factor o sumarle a todos una constante) claramente preservará las secuencias de juego ficticio. Cuando se varía el tamaño del juego, en cambio, la preservación puede no ser trivial o no valer, según la clase de juego.

Pero la intención es, además, que tras el cambio se mantenga la mayoría de los valores presentes en la matriz original. El caso general puede ser complejo, pero a priori damos la siguiente formalización. Llamaremos {\em rango de imagen} de una matriz $A$ al cardinal del conjunto de valores tomados por todos sus elementos. Buscamos entonces de ser posible ejemplos de juegos que tengan rango de imagen mínimo manteniendo las propiedades deseadas. En todos los casos nos interesará, , tras el agregado de filas o columnas, no la exactitud del rango de imagen sino el orden de magnitud en función de las dimensiones.

Diremos que una secuencia de juego ficticio $s$ termina con una secuencia $t$ (siendo esta finita o no), o que $t$ es un sufijo de $s$, si existe $s'$ finita tal que $s = s't$.

\begin{lemma}\label{preservación}
Sea $A = (a_{i,j})$ un juego de intereses idénticos de tamaño $n\times m$, y sean $n'\geq n$, $m'\geq m$. Entonces existe un juego de intereses idénticos $A'$ de tamaño $n'\times m'$ y con rango de imagen $O(n+m)$ tal que:
\begin{enumerate}
\item Toda secuencia $(i^\tau, j^\tau)_{\tau \in \mathbb{N}}$ de juego ficticio (simultáneo o alternante) en $A$ lo es también en $A'$.
\item Toda secuencia $(i^\tau, j^\tau)_{\tau \in \mathbb{N}}$ de juego ficticio (simultáneo o alternante) sobre $A'$ termina con una secuencia del mismo tipo sobre $A$.
\end{enumerate}
Además, si $A$ es no degenerado, puede pedirse que $A'$ también lo sea.
\end{lemma}

\begin{proof}%VERIFICAR
Sea $v = min(A)$, valor a utilizar en todo el proceso que sigue (es decir, no cambia a lo largo de las iteraciones).

Para obtener un juego de intereses idénticos $A'$ de $(n+1)\times m$ con la propiedad indicada, basta definir $a'_{n+1,j} = v - n - j$ para $1\leq j\leq m$. Entonces, si $i^\tau \in BR_1(y^{\tau-1})$ en $A$, también $i^\tau \in BR_1(y^{\tau-1})$ en $A'$, y similarmente para $j^\tau$ considerando $BR_2(x^{\tau-1})$ o $BR_2(x^\tau)$ (en los casos simultáneo o alternante, resp.) Se procede análogamente para obtener un juego de intereses idénticos de $n\times (m+1)$ definiendo $a'_{i,m+1} = v - m - i$ para $1\leq i\leq n$\footnote{Los términos $-j$ y $-i$ aseguran preservar la condición de no degenerado, requerida por el teorema \ref{teorema:afp:mejor}, y los términos $-n$ aseguran que finalmente el rango de imagen sea $O(n+m)$.}. Finalmente, se itera la construcción anterior $n'-n+m'-m$ veces para obtener $A'$ de un juego de intereses idénticos de $n'\times m'$.
\end{proof}

% La idea es que A' tenga un esquema como:
%
% v-1  v-2   v-3   ...  v-m
% v-2  v-3   v-4   ...  v-(m-1)
% ...
% v-n  v-n-1 v-n-2 ...  v-(n-m)
%
% pero con la A original ocupando el cuadrante superior izq.

En la demostración anterior, al agregar a la matriz una nueva fila, se reusan valores (pueden no ser necesarios tantos como $m$, el número de columnas). Asimismo, de no interesarnos reducir el rango de imagen, alcanzaría tan sólo con llenar cada nueva fila (o columna) con un valor como $a'_{n+1,j} = min(A) - j$ donde $j$ es el índice del elemento correspondiente\footnote{Hay otras maneras de ajustar el valor $v$ a fin de evitar espacios intermedios, es decir que los valores a agregar tiendan a ser enteros consecutivos, lo que dependerá naturalmente de los elementos presentes a lo largo de la matriz obtenida en el paso anterior, pero claramente no nos preocupa esto en tanto no afecte al rango de imagen. Por otra parte, si se ignorase el requerimiento de juego no degenerado, para todo $n,m$ se podría pedir que el rango de imagen no exceda en más que 1 al de la matriz $A$ original, simplificándose en mucho la demostración.}.

Nótese que la iteración se puede dar de distintas maneras, según si se consideran primero las filas y luego las columnas, o bien al revés, o bien con las distintas posibilidades de intercalado de estas con aquellas, resultando en cada caso una matriz diferente.

Veamos una propiedad similar para los juegos de suma cero.

\begin{lemma}\label{preservación2}
Sea $A = (a_{i,j})$ un juego de suma cero de tamaño $n\times m$, y sean $n'\geq n$, $m'\geq m$. Entonces existe un juego de suma cero $A'$ de tamaño $n'\times m'$ y con rango de imagen $O(n+m)$ tal que:
\begin{enumerate}
\item Toda secuencia $(i^\tau, j^\tau)_{\tau \in \mathbb{N}}$ de FP (simultáneo o alternante) en $A$ lo es también en $A'$.
\item Toda secuencia $(i^\tau, j^\tau)_{\tau \in \mathbb{N}}$ de FP (simultáneo o alternante) sobre $A'$ termina con una secuencia del mismo tipo sobre $A$.
\end{enumerate}
Además, si $A$ es no degenerado, puede pedirse que $A'$ también lo sea.
\end{lemma}

\begin{proof}%VERIFICAR
Sea $v = min(A)$, valor a utilizar en todo el proceso que sigue.
Para obtener un juego de suma cero $A'$ de $(n+1)\times m$ con la propiedad indicada, basta definir $a'_{n+1,j} = v - n - j$ para $1\leq j\leq m$. Entonces, si $i^\tau \in BR_1(y^{\tau-1})$ en $A$, también $i^\tau \in BR_1(y^{\tau-1})$ en $A'$, y similarmente para $j^\tau$ considerando $BR_2(x^{\tau-1})$ o $BR_2(x^\tau)$ (en los casos simultáneo o alternante, resp.) Para obtener un juego de suma cero de $n\times (m+1)$, se procede análogamente pero tomando de entrada $v = max(A)$ y luego definiendo $a'_{i,m+1} = v + m + i$ para $1\leq i\leq n$. Finalmente, se itera la construcción anterior $n'-n+m'-m$ veces para obtener $A'$ de un juego de suma cero de $n'\times m'$.
\end{proof}

Al igual que con el lema anterior, la iteración se puede dar de muchas maneras.

A priori, no contamos en cambio con un resultado similar para la reducción en tamaño de las matrices: puede haber secuencias que utilicen todas las acciones de cada jugador en $A'$ incluyendo las que no forman parte de $A$, de modo que no siempre será factible eliminar alguna estrategia en este último. 


\section{Velocidad de Convergencia de AFP} \label{sec:aportes:velocidad}

En esta sección nos enfocaremos en estudiar más detalladamente los resultados de Brandt, Fischer y Harrenstein \cite{brandt:rate:convergence}. En su paper, los autores presentan cotas superiores para la velocidad de convergencia del juego ficticio simultáneo en clases de juegos que se sabe que siempre convergen de forma pura \falta{checkear esto}. Como dijimos en el capítulo \ref{cap:relwork}, los autores mencionan la posibilidad de expandir sus resultados a la variante alternante de juego ficticio. Presentamos a continuación nuestro análisis para los casos de los juegos de suma constante simétricos y los no degenerados de intereses idénticos.
Por claridad, aprovecharemos el hecho de que todos los casos analizados en esta sección resultan en conjuntos de mejores respuestas unitarios para omitir las reglas de desempate, ya que no afectarán los resultados.

% Pero antes, nos parece oportuno hablar mas detalladamente sobre algunas cuestiones respecto a las métricas contra las que se estudia la velocidad de convergencia de un proceso de aprendizaje en juegos. Puede llamar la atención que si bien en toda la bibliografía previa al trabajo de Brandt, Fischer y Harrenstein se habla sobre la velocidad de convergencia en función del tamaño del juego, los autores prefieren estudiarla en función del tamaño de representación en bits de los valores de las utilidades del juego. Este tipo de resultados es en realidad más general. Observemos que un juego que tenga $k$ acciones para un jugador necesitará, en cualquier codificación capaz de representar juegos bimatriciales finitos arbitrarios, $O(k)$ bits, ya que deberá codificar cada una de estas $k$ acciones con al menos un bit. Por otro lado, a efectos del estudio de la convergencia del juego ficticio, un juego con conjuntos de acciones de tamaños menores a $k$ pero que requiere $O(k)$ bits para ser representado, puede expandirse a uno con comportamiento equivalente de tamaño $k$. Esto lo mostraremos en los siguientes lemas: \falta{hay que retocar este parrafo para referenciar la discusion previa sobre tamaños, bits, etc}

% \falta{comentar la cuestion de que se buscan caso especificos porque se generaliza a clases mas grandes}

Veamos primero que para el caso de los juegos de suma constante simétricos, el teorema de estos autores efectivamente es expandible de forma bastante directa a la variante alternante. Vale la pena notar que formulamos el teorema en términos de que el último perfil no sea un equilibrio de Nash puro, en vez de pedir que ninguno en la secuencia lo sea como hacen Brandt, Fischer y Harrenstein ya que por el principio de estabilidad, si alguno de los perfiles jugados en la secuencia fuera un equilibrio de Nash puro, todos los siguientes lo serían.

\begin{theorem} \label{teorema:afp:velocidad:simetricos}
    Existe un juego simétrico $A$ representable en $O(k)$ bits, con al menos un equilibrio de Nash puro y una secuencia de juego ficticio alternante $(i^\tau, j^\tau)_{\tau \in \mathbb{N}}$ sobre $A$ tal que $(i^{2^k}, j^{2^k})$ no es un equilibrio de Nash puro.
\end{theorem}
\begin{proof}
    Consideremos un juego en forma bimatricial con la siguiente matriz de pagos:

    \begin{center}
    \begin{tabular}{llll}
    \                           & $j_1$      & $j_2$      & $j_3$                            \\ \cline{2-4}
    \multicolumn{1}{l|}{$i_1$} & 0          & -1         & \multicolumn{1}{l|}{$-\epsilon$} \\
    \multicolumn{1}{l|}{$i_2$} & 1          & 0          & \multicolumn{1}{l|}{$-\epsilon$} \\
    \multicolumn{1}{l|}{$i_3$} & $\epsilon$ & $\epsilon$ & \multicolumn{1}{l|}{0}           \\ \cline{2-4}
    \end{tabular}
\end{center}

    Si $\epsilon < 1$, vemos que $(i^3, j^3)$ es el único equilibrio de Nash puro por ser el único perfil restante luego de realizar eliminación iterada de estrategias estrictamente dominadas. \falta{Esto no explicamos qué es. Supongo que a los conceptos previos?}

    Consideremos un número $k > 1$ arbitrario y sea $\epsilon = 2^{-k}$. Para estos valores, $\epsilon$ puede codificarse en $O(k)$ bits, mientras que las otras utilidades del juego son constantes, por lo que podemos afirmar que la representación del juego será tambien del orden de  $O(k)$ bits. Por lo tanto, si probamos que un proceso de juego ficticio alternante puede requerir $2^k$ rondas antes de que se juegue $(i^3, j^3)$, el teorema estará demostrado.

    Si el proceso comienza con el jugador fila jugando $i^1$, entonces las utilidades esperadas del jugador columna serán $-x^1A = (0, 1, \epsilon)$ y elegirá $j_2$.

    En la siguiente ronda, el jugador $1$ reaccionará con $i_3$, dando que el jugador $2$ tendrá una creencia sobre su estrategia de $x^2 = (\frac{1}{2},0,\frac{1}{2})$, por lo que las utilidades esperadas del jugador $2$ serán $-x^2A = (\frac{-\epsilon}{2}, \frac{1-\epsilon}{2}, \frac{\epsilon}{2})$y volverá a elegir $j_2$.

    Este perfil $(i_3, j_2)$ se repetirá $2^k - 1$ rondas, ya que tendremos que mientras $2 \le \tau \le 2^k$, se cumplirán:

    \begin{align*}
        x^\tau     &= \frac{\pstrat{i_1} + (\tau - 1) \pstrat{i_3}}{\tau} = (\frac{1}{\tau}, 0,\frac{\tau - 1}{\tau}) \\
        -x^{\tau}A &= (-\frac{(\tau-1) \epsilon}{\tau}, \frac{1-(\tau-1)\epsilon}{\tau}, \frac{\epsilon}{\tau}) \\
        y^\tau     &= \pstrat{j_2} = (0, 1, 0) \\
        Ay^\tau    &= (- 1, 0, \epsilon) \\
    \end{align*}

    \begin{table}
        \begin{tabular}{llllll}
    \toprule
    {} &       $(i^\tau, j^\tau)$ &             $x^\tau$ &               $x^{\tau}B$ &                $y^\tau$ &                 $Ay^\tau$ \\
    $\tau$ &                &                         &                           &                         &                           \\
    \midrule
    $1$         &  $(i_1,\ j_2)$ &  $(1,\ 0,\ 0)$                                   & $(0, 1, 2^{-k})$                                                                          &  $(0,\ 1,\ 0)$    &   $(-1,\ 0,\ 2^{-k})$ \\
    $2$         &  $(i_3,\ j_2)$ &  $(\frac{1}{2},\ 0,\ \frac{1}{2})$               & $(-2^{-(k+1)},\ \frac{1 - 2^{-k}}{2},\ 2^{-(k+1)})$                                        &  $(0,\ 1,\ 0)$    &   $(-1,\ 0,\ 2^{-k})$ \\
    $3$         &  $(i_3,\ j_2)$ &  $(\frac{1}{3},\ 0,\ \frac{2}{3})$               & $(-\frac{2^{-k+1}}{3},\ \frac{1 - 2^{-k+1}}{3},\ \frac{2^{-k}}{3})$                       &  $(0,\ 1,\ 0)$    &   $(-1,\ 0,\ 2^{-k})$ \\
                &  $\vdots$      &   &   &   &   \\
    $\tau$      &  $(i_3,\ j_2)$ &  $(\frac{1}{\tau},\ 0,\ \frac{\tau - 1}{\tau})$  & $(-\frac{(\tau-1)2^{-k}}{\tau},\ \frac{1 - (\tau-1)2^{-k}}{\tau},\ \frac{2^{-k}}{\tau})$  &  $(0,\ 1,\ 0)$    &   $(-1,\ 0,\ 2^{-k})$ \\
                &  $\vdots$      &   &   &   &   \\
    $2^k$       &  $(i_3,\ j_2)$ &  $(\epsilon,\ 0,\ \frac{\epsilon-1}{\epsilon})$           & $(\epsilon (\epsilon-1),\ \epsilon^2,\ \epsilon^2)$                                                     &  $(0,\ 1,\ 0)$    &   $(-1,\ 0,\ 2^{-k})$\\
    \bottomrule
    \\
    \end{tabular}
        \caption{Proceso de juego ficticio alternante en el juego del teorema \ref{teorema:afp:velocidad:simetricos}}
        \label{tabla:afp:velocidad:simetricos}
    \end{table}

    La tabla \ref{tabla:afp:velocidad:simetricos}, muestra como se desarrolla este proceso. Para el jugador fila, justificar su decisión es trivial ya que la estrategia percibida de su oponente es pura e $i_3$ es la única acción con utilidad esperada positiva. Para entender por qué el jugador columna no cambia su estrategia, debemos notar que:

    \begin{align*}
        \tau &\le 2^k = \frac{1}{\epsilon} \\
        \epsilon \tau&\le 1 \\
        \epsilon (\tau - 1 + 1) &\le 1\\
        (\tau - 1) \epsilon + \epsilon &\le 1\\
        1-(\tau-1)\epsilon &\ge \epsilon \\
        \frac{1-(\tau-1)\epsilon}{\tau} &\ge \frac{\epsilon}{\tau} \\
    \end{align*}

    Esto podemos interpretarlo como que si bien en las iteraciones recientes el jugador $1$ jugó $i_3$, el incentivo resultante de la única vez que jugó $i_1$ es muy fuerte por la gran diferencia de utilidades para el jugador $2$ entre $(i_1, j_2)$ e $(i_2, j_3)$, por lo que deberán pasar $2^{k-1}$ iteraciones de $i_3$ luego de ese único $i_1$ para que las utilidades esperadas se compensen.

    Concluimos entonces que la secuencia

    \begin{center}
    \begin{math}
        (i_1, j_2), \underbrace{(i_3, j_2), ... (i_3, j_2)}_{\text{$2^k - 1$ veces}}
    \end{math}
    \end{center}

    es una secuencia de aprendizaje de juego ficticio alternante válida de este juego que es exponencialmente larga en $k$ y en la cual no se juega ningún equilibrio de Nash puro.

\end{proof}

Por su parte, la demostración para el caso de los juegos no degenerados de $2 \times 3$ es un poco menos directa y requiere plantear una ligera variante del juego originalmente propuesto por Brandt, Fischer y Harrenstein. \falta{discutir un poco como afecta el cambio de la matriz}

\begin{theorem} \label{teorema:afp:velocidad:nondegen}
    Existe un juego no degenerado de intereses idénticos $A$ representable en $O(k)$ bits, con al menos un equilibrio de Nash puro y una secuencia de juego ficticio alternante $(i^\tau, j^\tau)_{\tau \in \mathbb{N}}$ sobre $A$ tal que para todo $\tau < 2^k$, $(i^\tau, j^\tau)$ no es un equilibrio de Nash puro.
\end{theorem}

\begin{proof}
    Consideremos un juego en forma bimatricial con la siguiente matriz de pagos:

    \begin{center}
    \begin{tabular}{llll}
    \                           & $j_1$    & $j_2$                      & $j_3$                           \\ \cline{2-4}
    \multicolumn{1}{l|}{$i_1$} & $1$ & $2$          & \multicolumn{1}{l|}{$0$} \\
    \multicolumn{1}{l|}{$i_2$} & $0$ & $2+\epsilon$ & \multicolumn{1}{l|}{$2+2\epsilon$} \\ \cline{2-4}
    \end{tabular}
\end{center}

    Si $\epsilon < 1$, vemos que $(i^2, j^3)$ es el único equilibrio de Nash puro por ser el único perfil restante luego de realizar eliminación iterada de estrategias estrictamente dominadas.

    Consideremos un número $k > 1$ arbitrario. Mostraremos que para $\epsilon = 2^{-k}$, un proceso de juego ficticio alternante puede tomar $2^k$ rondas antes de que se juegue $(i^2, j^3)$. Al igual que en la demostración anterior, el juego puede codificarse en $O(k)$ bits, por lo que esto demuestra el teorema.

    Si el proceso comienza con el jugador fila jugando $i^1$, entonces $x^1B = (1, 2, 0)$ y por lo tanto el jugador columna elegirá $j_2$.

    En la siguiente ronda, el jugador $1$ reaccionará con $i_2$, dando que el jugador $2$ tendrá una creencia sobre su estrategia de $x^2 = (\frac{1}{2},\frac{1}{2})$, por lo que las utilidades esperadas del jugador $2$ serán $-x^2A = (\frac{1}{2}, 2 + \frac{\epsilon}{2}, 1 + \epsilon)$ y volverá a elegir $j_2$.

    Este perfil $(i_2, j_2)$ se repetirá $2^k - 1$ rondas, ya que tendremos que mientras $2 \le \tau \le 2^k$, se cumplirán:

    \begin{align*}
        x^\tau     &= \frac{\pstrat{i_1} + (\tau - 1) \pstrat{i_2}}{\tau} = (\frac{1}{\tau}, \frac{\tau - 1}{\tau}) \\
        x^{\tau}A &= (\frac{1}{\tau}, \frac{2 + (\tau - 1)(2 + \epsilon)}{\tau}, \frac{(\tau - 1)(2 + 2 \epsilon)}{\tau}) \\
        y^\tau     &= \pstrat{j_2} = (0, 1, 0) \\
        Ay^\tau    &= (2, 2 + \epsilon) \\
    \end{align*}

    \begin{table}
        \begin{tabular}{llllll}
\toprule
{} &       $(i, j)$ &              $x$ &                    $x{^\tau}B$ &                     $y$ &             $Ay$ \\
Ronda   &                &                  &                         &                         &                  \\
\midrule
1       &  $(i_1,\ j_2)$ &  $(1, 0)$                                    &  $(1, 2, 0)$                                                                                              &  $(0, 1, 0)$ &  $(2, 2 + \epsilon)$ \\
2       &  $(i_2,\ j_2)$ &  $(\frac{1}{2}, \frac{1}{2})$                &  $(\frac{1}{2}, 2 + \frac{\epsilon}{2}, 1 + \epsilon)$                                                    & $(0, 1, 0)$   & $(2, 2 + \epsilon)$ \\
3       &  $(i_2,\ j_2)$ &  $(\frac{1}{3}, \frac{2}{3})$                &  $(\frac{1}{3}, 2 + \frac{2}{3} \epsilon, \frac{4 + 4 \epsilon)}{3})$   & $(0, 1, 0)$   & $(2, 2 + \epsilon)$ \\
        &  \vdots       \\
$\tau$  &  $(i_2,\ j_2)$ &  $(\frac{1}{\tau}, \frac{\tau - 1}{\tau})$   &  $(\frac{1}{\tau}, \frac{2 + (\tau - 1)(2 + \epsilon)}{\tau}, \frac{(\tau - 1)(2 + 2 \epsilon)}{\tau})$   & $(0, 1, 0)$   & $(2, 2 + \epsilon)$ \\
        &  \vdots       \\
$2^k$   &  $(i_2,\ j_2)$ &  $(\epsilon, 1 - \epsilon)$   &  $(\epsilon, 1 - \epsilon - \epsilon^2, 2 - \epsilon^2)$   & $(0, 1, 0)$   & $(2, 2 + \epsilon)$ \\
\bottomrule
\end{tabular}

        \caption{Proceso de juego ficticio alternante en el juego del teorema \ref{teorema:afp:velocidad:nondegen}}
        \label{tabla:afp:velocidad:nondegen}
    \end{table}

    La tabla \ref{tabla:afp:velocidad:nondegen}, muestra como se desarrolla este proceso. Para el jugador fila, justificar su decisión es trivial ya que la estrategia percibida de su oponente es pura y la utilidad esperada de  $i_2$ es siempre marginalmente mayor que la de $i_1$. Para entender por qué el jugador columna no cambia su estrategia, debemos notar que:

    \begin{align*}
        \tau &\le 2^k = \frac{1}{\epsilon} \\
        \tau &\le \frac{2}{\epsilon} + 1 \\
        \frac{2}{\epsilon} &\ge \tau - 1 \\
        \frac{2}{\tau - 1} &\ge \epsilon \\
        \frac{2}{(\tau - 1)} + 2 + \epsilon &\ge 2 + 2 \epsilon \\
        2 + (\tau - 1) (2 + \epsilon) &\ge (\tau - 1) (2 + 2 \epsilon) \\
    \end{align*}

    Similarmente al teorema anterior, esto podemos interpretarlo como que si bien en las iteraciones recientes el jugador $1$ jugó $i_2$, el incentivo resultante de la única vez que jugó $i_1$ es muy fuerte por la gran diferencia de utilidades para el jugador $2$ entre $(i_1, j_2)$ e $(i_2, j_3)$ \falta{checkear esto}, por lo que deberán pasar $2^{k-1}$ iteraciones de $i_3$ luego de ese único $i_1$ para que las utilidades esperadas se compensen.

    Concluimos entonces que la secuencia

    \begin{center}
    \begin{math}
        (i_1, j_2), \underbrace{(i_2, j_2), ... (i_2, j_2)}_{\text{$2^k - 1$ veces}}
    \end{math}
    \end{center}

    es una secuencia de aprendizaje de juego ficticio alternante válida de este juego que es exponencialmente larga en $k$ y en la cual no se juega ningún equilibrio de Nash puro.
\end{proof}

El detalle de que el juego originalmente propuesto por Brandt, Fischer y Harrenstein no nos sirva para demostrar el teorema anterior no es para nada menor. En efecto, como veremos en el siguiente teorema, es un ejemplo de un juego en el que un proceso de juego ficticio simultáneo puede requerir una cantidad de rondas exponenciales mientras que, toda secuencia de juego ficticio alternante convergerá rápidamente.

\begin{theorem} \label{teorema:afp:mejor}
    Existe un juego no degenerado de intereses idénticos $A$ representable en $O(k)$ bits y una secuencia de juego ficticio simultáneo $(i^\tau, j^\tau)_{\tau \in \mathbb{N}}$ sobre $A$ y una constante $C$ tal que
    \begin{itemize}
        \item Para todo $\tau < 2^k$, $(i^\tau, j^\tau)$ no es un equilibrio de Nash puro.
        \item Para todo proceso de juego ficticio alternante $(\widehat{i}^\tau, \widehat{j}^\tau)_{\tau \in \mathbb{N}}$ en $A$, $(\widehat{i}^C, \widehat{j}^C)$ es un equilibrio de Nash puro.
    \end{itemize}
\end{theorem}
\begin{proof}
    Consideremos el juego en forma bimatricial con la siguiente matriz de pagos:

    \begin{center}
    \begin{tabular}{llll}
    .                          & $j^1$    & $j^2$                      & $j^3$                           \\ \cline{2-4}
    \multicolumn{1}{l|}{$i^1$} & $1$ & $2$                   & \multicolumn{1}{l|}{$0$} \\
    \multicolumn{1}{l|}{$i^2$} & $0$ & $2+\epsilon$ & \multicolumn{1}{l|}{$3$} \\ \cline{2-4}
    \end{tabular}
\end{center}

    Si $\epsilon < 1$, $(i^2, j^3)$ es el único equilibrio de Nash puro por ser el único perfil restante luego de realizar eliminación iterada de estrategias estrictamente dominadas.

    Consideremos un número $k > 1$ arbitrario. Mostraremos que para $\epsilon = 2^{-k}$, un proceso de juego ficticio simultáneo puede tomar $2^k$ rondas antes de que se juegue $(i^2, j^3)$, mientras que todo proceso de juego ficticio alternante converge de forma pura en un número constante de rondas. Al igual que los teoremas anteriores, el juego puede codificarse en $O(k)$ bits, por lo que esto demuestra el teorema.

    Veamos primero el caso alternante. Existen dos posibles secuencias de juego ficticio alternante para este juego, dado que el jugador $1$ elegirá primero y el jugador $2$ reaccionara según esta decisión.

    \begin{table}
        \centering
        \begin{tabular}{llllll}
    \toprule
    {} &       $(i, j)$ &              $x$ &                    $xB$ &                     $y$ &             $Ay$ \\
    Iteración &                &                  &                         &                         &                  \\
    \midrule
    $1$         &  $(i_2, j_3)$ &  $(0, 1)$ &  $(0, 2 + \epsilon, 3)$ &  $(0, 0, 1)$ &  $(0 , 3)$\\
    \bottomrule
    \\
    \end{tabular}
        \caption{Proceso de juego ficticio alternante sobre el juego del teorema \ref{teorema:afp:mejor} comenzando por $i_2$}
        \label{tabla:afp:mejor:b}
        \centering
        \begin{tabular}{llllll}
    \toprule
    {} &       $(i^\tau, j^\tau)$ &             $x^\tau$ &               $x^{\tau}B$ &                $y^\tau$ &                 $Ay^\tau$ \\
    $\tau$ &                &                         &                           &                         &                           \\
    \midrule
    $1$         &  $(i_1,\ j_2)$ &  $(1, 0)$ &  $(1, 2, 0 )$ &  $(0, 1, 0)$ &  $(2, 2 + \epsilon)$ \\
    $2$         &  $(i_2,\ j_2)$ &  $(\frac{1}{2}, \frac{1}{2})$ &  $(\frac{1}{2}, 2 + \frac{\epsilon}{2}, \frac{3}{2})$ &  $(0, 1, 0)$ &  $(2, 2 + \epsilon)$ \\
    $3$         &  $(i_2,\ j_2)$ &  $(\frac{1}{3}, \frac{2}{3})$ &  $(\frac{1}{3}, 2 + \frac{2}{3}\epsilon, 2)$ &  $(0, 1, 0)$ &  $(2, 2 + \epsilon)$ \\
    $4$         &  $(i_2,\ j_3)$ &  $(\frac{1}{4}, \frac{3}{4})$ &  $(\frac{1}{4}, 2 + \frac{3}{4}\epsilon, \frac{9}{4})$ &  $(0, \frac{3}{4}, \frac{1}{4})$ &  $(\frac{3}{2}, \frac{3(3+\epsilon)}{4})$\\
    \bottomrule
    \\
    \end{tabular}
        \caption{Proceso de juego ficticio alternante sobre el juego del teorema \ref{teorema:afp:mejor} comenzando por $i_1$}
        \label{tabla:afp:mejor:a}
        \centering
        \begin{tabular}{llllll}
\toprule
{} &       $(i, j)$ &              $x$     $x{^\tau}B$ &    $y$ &             $Ay$ \\
Iteración &                &                  &                         &                         &                  \\
\midrule
$1$         &  $(i_1, j_1)$ &  $(1, 0)$ &  $(1, 2, 0)$ &  $(1, 0, 0)$ &                                $(1 , 0)$ \\
$2$         &  $(i_1, j_2)$ &  $(1, 0)$ &  $(1, 2, 0)$ &  $(\frac{1}{2}, \frac{1}{2}, 0)$ &              $(\frac{3}{2}, \frac{2 + \epsilon}{2})$ \\
$3$         &  $(i_1, j_2)$ &  $(1, 0)$ &  $(1, 2, 0)$ &  $(\frac{1}{3}, \frac{2}{3}, 0)$ &              $(0 , 3)$ \\
        &  \vdots       \\
$\tau$  &  $(i_1,\ j_2)$    &  $(1, 0)$ &  $(1, 2, 0)$  & $(\frac{1}{\tau}, \frac{\tau - 1}{\tau}, 0)$ & $(2 - \frac{1}{\tau}, 2 - \frac{2 + \epsilon}{\tau})$ \\
        &  \vdots       \\
$2^k$   &  $(i_1,\ j_2)$ &     $(1, 0)$ &  $(1, 2, 0)$  & $(\frac{1}{2^k}, \frac{2^k-1}{2^k}, 0)$      & $(2 - \frac{1}{2^k}, 2 - \frac{2 + \epsilon}{2^k})$ \\
            \bottomrule
\end{tabular}

        \caption{Proceso de juego ficticio simultáneo sobre el juego del teorema \ref{teorema:afp:mejor} comenzando por $(i_1, j_1)$}
        \label{tabla:afp:mejor:c}
    \end{table}

    Si el jugador fila juega $i^2$, el jugador columna responderá con $j^3$, siendo este el equilibrio puro. Si el jugador fila comienza con $i^1$, el jugador columna responderá con $j^2$. Esto hará que el jugador fila juegue $i^2$ en la segunda ronda por ser $A\pstrat{j^2} = (2, 2 + \epsilon)$, mientras que el jugador continuará fila continuara jugando $j^2$. Esta situación se repetirá una ronda más, tras la cuál, la jugador columna se verá incentivado a jugar $j^3$. Los desarrollos de estos dos procesos pueden verse en las tablas \ref{tabla:afp:mejor:a} y \ref{tabla:afp:mejor:b}. Vemos entonces que, independientemente del valor de $k$, ambos procesos convergen de forma pura en 4 rondas o menos.

    Pasemos ahora al caso simultáneo, para el cual nos basaremos en la prueba de Brandt, Fischer Y Harrenstein. Si el proceso comienza con el perfil $(i^1, j^1)$, entonces las utilidades esperadas serán $Ay^1 = (1, 0)$ e $x^1 B = (1, 2, 0)$ respectivamente. Luego, en la segunda iteración el jugador fila elegirá $i^1$ y el jugador columna $j^2$. Las utilidades esperadas se actualizarán entonces como $Ay^2 = (\frac{3}{2}, \frac{2 + \epsilon}{2})$ y $x^2 B = (1, 2, 0)$.

    A continuación, por al menos $2^k$ rondas, los jugadores elegirán las mismas jugadas que en la iteración 2, dado que para todo $\tau$ tal que $2 \le i \le 2^k$, tendremos $Ay^\tau = (2 - \frac{1}{\tau}, 2 - \frac{2 + \epsilon}{\tau})$ e $x^\tau B = (1, 2, 0)$. La tabla \ref{tabla:afp:mejor:c} muestra como se desarrolla este proceso.

    Concluimos entonces que la secuencia de perfiles

    \begin{center}
    \begin{math}
        (i_1, j_1), \underbrace{(i_1, j_2), ... (i_1, j_2)}_{\text{$2^k$ veces}}
    \end{math}
    \end{center}


    es una secuencia de aprendizaje de juego ficticio simultáneo válida de este juego que es exponencialmente larga en $k$ y en la cual no se juega ningún equilibrio de Nash puro.

\end{proof}