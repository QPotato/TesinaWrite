\documentclass{article}
\usepackage[utf8]{inputenc}
\begin{document}
\section*{Versiones}

\subsection*{2021-09-27}
\begin{itemize}
    \item Primera entrega.
\end{itemize}


\subsection*{2021-10-12}
\begin{itemize}
    \item Expansión de las conclusiones.
    \item Mención al trabajo presentado en el SIIIO en el marco de la tesina.
\end{itemize}

\subsection*{2022-2-10}
\begin{itemize}
    \item En el resumen, último párrafo, se cambia "...que indican que esta podría tener una utilidad práctica superior a su contraparte simultánea como mecanismo para encontrar equilibrios de Nash" por "..que esta podría ser, desde un punto de vista computacional, un mecanismo al menos tan eficiente como el simultáneo para encontrar equilibrios de Nash, e incluso mejor en algunos casos.".
    \item En el capítulo 1, página 1, se cambia "Para ello, la teorı́a estudia los comportamientos, la posibilidad de maximizar ganancias según variados criterios y los
    casos de equilibrio, a la vez que formula y analiza modelos." por "Para ello, la teoría estudia los comportamientos, la posibilidad de maximizar ganancias según variados criterios y, a través de diversos conceptos de equilibrios, formula y analiza modelos."
    \item En el capítulo 1, página 1, se cambia "El enfoque clásico trata sobre juegos en forma normal..." por "El enfoque clásico, en un contexto no cooperativo, trata sobre juegos en forma normal...".
    \item Se agregan mas referencias en el capítulo 1.
    \item En el capítulo 1, página 2, se cambia "Este segundo enfoque es un poco contra-intuitivo, pues rompe uno de los principios de la teoría de juegos: la toma de decisiones de cada jugador se considera siempre como procesos independientes uno del otro." por "Este segundo enfoque es un poco contra-intuitivo, pues una interpretación común de los juegos estratégicos es que la toma de decisiones de cada jugador son procesos independientes uno del otro."
    \item En la sección 1.1, página 3, se cambia "Además presentaremos un lema sobre la conservación del juego ficticio al expandir juegos, que nos permitirá generalizar los resultados." por "Asimismo, una prueba de la conservación del juego ficticio al expandir juegos, nos permitirá generalizar los teoremas de velocidad de convergencia."
    \item En la sección 2.1, página 6, se cambia "...método para computar equilibrios de Nash..." por "...método para calcular equilibrios de Nash...".
    \item En la sección 3.1, página 7, se cambia "Llamaremos estrategias mixtas del jugador 1 a las distribuciones de probabilidades sobre su conjunto de acciones $N$ y las representaremos con vectores fila $x \in \Delta(N)$ de tamaño $n$. Similarmente, las estrategias mixtas del jugador 2 serán distribuciones de probabilidad sobre $M$ que notaremos con vectores columna $y \in \Delta(M)$ de tamaño $m$." por "Llamaremos \emph{estrategias mixtas} del jugador 1 a cada elemento del espacio de probabilidad sobre su conjunto de acciones $N$, notado $\Delta(N)$, y las representaremos con vectores fila $x$ de tamaño $n$. Similarmente, las estrategias mixtas del jugador 2 serán elementos de $\Delta(M)$ que representaremos con vectores columna $y$ de tamaño $m$.".
    \item En la sección 3.1, página 8, se quita la frase "Uno de los teoremas fundacionales del área, conocido como el Teorema de Minimax [26] establece que todos poseen al menos un equilibrio de Nash puro y a la ganancia para el jugador fila en un equilibrio la llamamos valor del juego (pues es la misma en todos los equilibrios)."
    \item En la sección 3.2, página 8, se corrige la definición de juego simétrico.
    \item En la sección 3.3, página 9, se cambia "Algunos autores como Berger, Monderer, Sela, Shapley, Daskalakis y Pan [6][8][11][13]" por "Algunos trabajos como Berger [6], Monderer y Sela [8], Monderer y Shapley [11], Daskalakis y Pan [13]"
    \item En sección 3.3, página 10, se cambia "conjunto de menor respuesta" por "conjunto de mejor respuesta".
    \item En sección 3.3, página 10, se cambia "...normalmente en teoría de juegos se representa jugadores eligiendo simultánea e independientemente." por "normalmente, en el contexto de juegos no cooperativos, se representa jugadores eligiendo simultánea e independientemente.".
    \item Las definiciones 3.3.2 y 3.3.3, páginas 9 y 10, se cambian para dejar más claro el concepto de secuencia de juego ficticio como un elemento de un proceso de juego ficticio.
    \item En definiciones 3.3.3, página 10, se corrigen los índices para incluir los casos de $p^1$ y $q^1$.
    \item En al definición 3.4.2, página 11, se cambia "... si existe un equilibrio de Nash mixto tal que... " por "si existe un equilibrio de Nash mixto $(x^*, y^*)$ tal que ...".
    \item En la sección 4.1, página 17, se cambia "La idea será probar que los historiales..." por "La idea será probar que los contadores...".
    \item En la tabla 4.1 se corrige el valor de $x$ en la fila $2^k$.
    \item En la demostración del lema 4.1.1, página 17, se corrigen errores menores.
    \item Sobre el final del tercer párrafo del lema 4.2.1, página 19, se corrige un signo mayor por mayor o igual.
    \item El teorema 4.2.1 se expresa en términos de $\tau$ en lugar de $k$.
    \item En la sección 4.2, página 19, se cambia "Este principio nos será útil en el capítulo siguiente..." por "Este principio nos será útil en la sección 4.4 ...".
    \item En el lema 4.2.2, página 20 se cambia "juego ficticio simultáneo" por "juego ficticio alternante".
    \item En la sección 4.3, página 22, se cambia "En todos los casos nos interesará, tras el agregado de filas o columnas, no la exactitud del rango de imagen sino el orden de magnitud en función de las dimensiones." por "En todos los casos nos interesará, tras el agregado de filas o columnas, el orden de magnitud del rango de imagen en función de las dimensiones.".
    \item En la demostración del lema 4.3.1, página 24, se cambia "aseguran un rango de imagen $O(n+m)$." por "aseguran un rango de imagen $O(n'+m')$.".
    \item En la sección 4.4, página 24, se cambia "...para los casos de los juegos simétricos de suma cero y los no degenerados de intereses idénticos." por "...para los casos de los juegos simétricos y los no degenerados de intereses idénticos".
    \item En la sección 4.4, en las páginas 24 y 26, se aclaran mejor las referencias a los teoremas de [3].
    \item En las demostraciones de los teoremas 4.4.2 y 4.4.3 se mueve el $\epsilon < 1$ del primer párrafo al planteo de la matriz.
    \item Al final de las demostraciones de los teoremas 4.4.2 y 4.4.3, se cambia "...exponencialmente larga en $k$ en la que no aparece ningún equilibrio de Nash puro, y el teorema sigue..." por "...exponencialmente larga en $k$ en la que no aparece ningún equilibrio de Nash puro para el juego de $2 \times 3$. El teorema para el caso de $n \times m$ sigue...".
    \item En el capítulo 5, página 31, se cambia "...la toma simultánea e independiente de decisiones es un axioma fundacional de la teoría de juegos (precisamente de los juegos en forma normal)." por "...la toma simultánea e independiente de decisiones es un supuesto muy común de la teoría de juegos (precisamente de los juegos no cooperativos).".
    \item A lo largo del trabajo, se reemplazan símbolos de cuantificación por las palabras "para todo" y símbolos de implicancia por las palabras "implica que".
    \item Se cambia el orden de la bibliografía a por fecha.
    \item Se lleva toda la bibliografía al mismo formato.
    \item Correcciones generales de ortografía.
\end{itemize}
\end{document}

